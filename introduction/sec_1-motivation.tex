\section{Motivation}
To enhance the accuracy of weather predictions, it is necessary to provide reliable and detailed weather data as inputs to \acrfull{nwm}. These NWMs utilize \db{multimodal} weather data collected via diverse sources such as automated weather stations, radars, air balloons, and satellite images. Prior to feeding such diverse data (collected from different sources that belongs to different stakeholders) into respective NWMs, it is necessary to integrate data into a common format. Moreover, the data integration system’s response time need to be relatively low to %accommodate critical situations 
\db{forecast and monitor time-sensitive weather events} like hurricanes, storms, and floods which require rapid and frequent execution of NWMs.

\dbc{We are not developing this platform only because we need to have a common data format or need to share with 3rd parties. These are secondary reasons. Highlight key reason why we need a weather data assimilation system. Key reason should be indicated in Abstract too.}

Providing public to access weather data is also useful to enable many third-party applications and research. For example, logistic companies could use those data with their models to plan and schedule their deliveries. Agricultural insurance companies can warn the farmers in advance, as well as calculate premiums based on anticipated weather patterns.
\dbc{Highlight why we need a platform to provide such 3rd party access. Should be indicated in Abstract too.}

While Data Integration and Analysis System (DIAS) and Meteorological Assimilation Data Ingest System (MADIS) are some of well-known Weather Data Integration and Assimilation (WDIA) systems, they are proprietary. Further Delft-FEWS is free to use, it is not open source. Hence, users of WDIA are forced to pay heavy licenses and are unable to extend the solutions to cater to their country-specific requirements. Therefore, the objective of this research is to develop a WDIA system for Sri Lanka, as well as make it open source, so that others could user and contribute to the solution.
\dbc{Saying something proprietary is not good reason to do research. You should highlight the context under which these solutions are developed and differences between CUrW and Sri Lanka in general. Also, give a bit more details than saying "to cater to their country-specific requirements". All key systems such as DIAS, MADIS, Delf, etc. need to be properly cited.}
