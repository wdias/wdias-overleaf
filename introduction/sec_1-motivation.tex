Weather prediction is essential to reduce impact due to natural disasters and to gain effectively manage natural resources like water. The weather predictions/forecasts are performed using \acrfull{nwm} that use a set of configuration parameters and the current status of the environment such as temperature, rainfall, and wind direction as the input. For example, the \acrfull{wrf} \cite{MesoscaleMicroscaleMeteorologyLaboratoryWeatherModel} model is used to predict the rainfall, temperature, and wind direction. Typically multiple \acrshort{nwm} are executed on the same data with different configuration parameters. Then using the real-time observations, the correct model is selected. After removing the bias, the extract information is then fed into another set of models/simulators. 
As an example, FLO2D is a hydrologic model that can use to predict the surface water level. A given FLO2D model can simulate with the initial surface water levels and the sequence of forecasted rainfall for the corresponding modeled area from the \acrshort{nwm} model to forecast future water levels.
\dbc{Highighted part doesn't make much sense to me. Need to be reworded.}
\dbc{cite both NWM and WRF}
\gkc{FIXED BOTH}

\acrfull{curw} \cite{CUrWSL2017SL} is tasked with the mission to reduce water-related natural disasters in Sri Lanka. \acrshort{curw} \cite{CUrWSLObservedSL} is implementing its system with the idea of long-term maintainability rather depend on existing weather data integration systems. It collected real-time data from weather stations and other government entities and generates daily forecasts. Then the generated results are shared with relevant public departments responsible for disaster planning, mitigation, and management. One of the future goals of \acrshort{curw} is to open up data to the public enabling both societal and commercial use cases.

%%%%%%%%%%%%%%%%%%%%%%%%%%%%%%%%%%%%%%%%%%%%%%%%%%%%%%%%%%%%%%%%%%%%%%%%%%%%%%%%
\section{Motivation}
To enhance the accuracy of weather predictions, it is necessary to provide reliable and detailed weather data as inputs to \acrshort{nwm}. These NWMs utilize multi-modal weather data collected via diverse sources such as automated weather stations, radars, air balloons, and satellite images. Before feeding such diverse data (collected from different sources that belong to different stakeholders) into respective NWMs, it is necessary to integrate data into a common format. Moreover, the data integration system’s response time needs to be relatively low to forecast and monitor time-sensitive weather events like hurricanes, storms, and floods which require rapid and frequent execution of NWMs.

Processed weather data needs to be accessed by many third-party applications and research to make use of them. For example, logistics companies could use those data with their models to plan and schedule their deliveries. Agricultural insurance companies can warn the farmers, as well as calculate premiums based on anticipated weather patterns. Also, farmers rely on weather forecasts to decide what work to do on any particular day. For example, drying hay is only feasible in dry weather. Further, the aviation industry is especially sensitive to the weather, accurate weather forecasting is essential since before takeoff on the conditions to expect en route and at their destination. 
Thus it is required a platform to provide such 3rd party access, and use the data or process those based on their applications.

While Data Integration and Analysis System (DIAS) \cite{Kawasaki2018DataReduction} and Meteorological Assimilation Data Ingest System (MADIS) \cite{Macdermaid2005ARCHITECTUREP2.39} are some of well-known Weather Data Integration and Assimilation (WDIA) systems, they are proprietary. Further \acrfull{fews} \cite{Werner2013TheSystem} is free to use, it is not open source.
Even though there are many data integration systems, most of them are proprietary or closed source. Which means for an island like Sri Lanka that having different kinds of weather seasons over the year than that software originated need to be highly customized. Throughout the operations and maintenance, it is required to get support with paying lots of charges, and difficult to debug into a closed source system without maintenance support from the vendors. As an example, \acrshort{fews} does not support FLO2D model integration, and \acrshort{curw} is using FLO2D for water level forecasting in the data flow. In such a case, it required to get support from the vendor to integrate and configure the system.
And most of the software existing out there is not up to date with the latest technology concepts such as Cloud Computing. \acrfull{wdias} uses the modern architecture pattern such as microservice to achieve scalability, rather using monolithic architecture or client service architecture. Most existing systems required to shut down to update the system configuration such as \acrshort{fews}, but a system that is accessed by any third parties should be operating 99.9\% without any downtime and should allow adding new features on the fly with rollover capabilities. The proposed system should be independent of the platform. But most software out there is not, as an example \acrshort{fews} can run on Windows and Linux. And should be independent underline infrastructure providers which gives much flexibility for the users to decide the best cost-effective way to host it. Given the above facts, there is a necessity for such a system to be implemented.

\dbc{Citations need to be given for DIAS, WDIA, FEWS, FLOW2D, etc. By the way, don't remove my comments (you are welcome to comment on them). I need them to check whether those are addressed. When I see them addressed, I'll remove comments.}
\gkc{FIXED: Added citations}