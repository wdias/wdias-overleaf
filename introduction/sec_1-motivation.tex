Weather forecast is essential to reduce the impact caused by natural disasters and to effectively manage natural resources like water.
The weather forecasts are performed using \acrfull{nwms} that use a set of configuration parameters and the current status of the environment such as temperature, rainfall, and wind direction as the inputs. For example, the \acrfull{wrf} \cite{MesoscaleMicroscaleMeteorologyLaboratoryWeatherModel} model is used to predict the rainfall, temperature, and wind direction. Typically multiple \acrshort{nwms} are executed on the same data with different configuration parameters. Then using the real-time observations, the most fitting model is selected. After removing the bias, the data extracted from a \acrshort{nwm} are then fed into another set of models/simulators. For example, FLO2D, a hydrologic model that predicts surface water level, could use precipitation estimates from \acrshort{wrf} to forecast flood levels.

\acrfull{curw} \cite{CUrWSL2017Lanka} is tasked with the mission to reduce water-related natural disasters in Sri Lanka. 
However, as the existing weather data integration systems are not open source and not flexible enough to change without much help from the maintainers, the \acrshort{curw} \cite{CUrWSLObservedSL} is implementing own system with the idea of long-term maintainability.
\acrshort{curw} collects real-time data from weather stations and other relevant entities and generates daily forecasts. Then the generated results are shared with relevant public departments responsible for disaster planning, risks mitigation, and disaster management. One of the future goals of \acrshort{curw} is to open up data to the public enabling both societal and commercial use cases.

%%%%%%%%%%%%%%%%%%%%%%%%%%%%%%%%%%%%%%%%%%%%%%%%%%%%%%%%%%%%%%%%%%%%%%%%%%%%%%%%
\section{Motivation}
To enhance the accuracy of weather forecast, it is necessary to provide reliable and detailed weather data as inputs to \acrshort{nwm}. These \acrshort{nwms} utilize multi-modal weather data collected via diverse sources such as automated weather stations, radars, air balloons, and satellite images. Before feeding such diverse data (collected from different sources that belong to different stakeholders) into respective \acrshort{nwms}, it is necessary to convert them to a data format that can be ingest by the models. Moreover, the data integration system's response time needs to be relatively low to forecast and monitor time-sensitive weather events like hurricanes, storms, and floods which require rapid and frequent execution of \acrshort{nwms}.

Weather data processed by a \acrshort{nwm} maybe used by several other \acrshort{nwms}, as well as by many third-party applications and researcher. For example, logistic companies could use the processed data with their data models to plan and schedule their deliveries. Agricultural insurance companies can warn the farmers, as well as calculate premiums based on the anticipated weather patterns. Besides, farmers could rely on weather forecasts to plan their work.

\acrfull{dias} \cite{Kawasaki2018DataReduction} and \acrfull{madis} \cite{Macdermaid2005ArchitectureP2.39} are two of the well-known Weather Data Integration and Assimilation (WDIA) systems. However, those are proprietary systems. Further, while \acrfull{fews} \cite{Werner2013TheSystem} is free to use, but it is not open source. While a few other data integration systems exists, most of them are also proprietary or closed source. Thus, it is difficult and costly to customize such systems for an island like Sri Lanka which experience different kinds of weather seasons over the year \cite{NaveendrakumarFiveLanka}.
For example, \acrshort{fews} does not support FLO2D model integration, which is used by \acrshort{curw} to forecast water level.
To address such issues, there is also an initiative to implement an open-data open-model framework called Meta Scientific Modeling (MSM). 
But the existing WDIA systems and emerging systems are also based on the monolithic or client service architecture; hence, cannot benefit from technologies such as cloud computing to achieve high scalability and availability. Moreover, such systems cannot be updated or reconfigured without taking the system off line. Furthermore, these systems are platform specific. For example, while \acrshort{fews} can run on either Windows or Linux, downtime is needed to introduce any changes to the system. 

Given the above reasons, there is a necessity to develop a WDIA system that could handle large volumes of multimodel data efficiently while providing scalability and high throughput. Moreover, it is desirable to support controlled data sharing with other \acrshort{nwms} and third-party access to enable multiple use cases. Furthermore, such a system should be architectured to benefit from technologies such as cloud computing, microservices, and loosely coupled containerized applications.
