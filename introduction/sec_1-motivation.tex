Weather forecast is essential to reduce the impact caused by natural disasters and to effectively manage natural resources like water.
The weather forecasts are performed using \acrfull{nwm}s that use a set of configuration parameters and the current status of the environment such as temperature, rainfall, and wind direction as the inputs. For example, the \acrfull{wrf} \cite{MesoscaleMicroscaleMeteorologyLaboratoryWeatherModel} model is used to predict the rainfall, temperature, and wind direction. Typically multiple \acrshort{nwm} are executed on the same data with different configuration parameters. Then using the real-time observations, the most fitting model is selected. After removing the bias, the data extracted from a \acrshort{nwm} are then fed into another set of models/simulators. For example, FLO2D, a hydrologic model that predicts surface water level, could use precipitation estimates from \acrshort{wrf} to forecast flood levels.

\acrfull{curw} \cite{CUrWSL2017SL} is tasked with the mission to reduce water-related natural disasters in Sri Lanka. \acrshort{curw} \cite{CUrWSLObservedSL} is implementing own system with the idea of long-term maintainability rather than depending on existing weather data integration systems. \acrshort{curw} collects real-time data from weather stations and other relevant entities and generates daily forecasts. Then the generated results are shared with relevant public departments responsible for disaster planning, risks mitigation, and disaster management. One of the future goals of \acrshort{curw} is to open up data to the public enabling both societal and commercial use cases.

%%%%%%%%%%%%%%%%%%%%%%%%%%%%%%%%%%%%%%%%%%%%%%%%%%%%%%%%%%%%%%%%%%%%%%%%%%%%%%%%
\section{Motivation}
To enhance the accuracy of weather forecast, it is necessary to provide reliable and detailed weather data as inputs to \acrshort{nwm}. These NWMs utilize multi-modal weather data collected via diverse sources such as automated weather stations, radars, air balloons, and satellite images. Before feeding such diverse data (collected from different sources that belong to different stakeholders) into respective NWMs, it is necessary to convert them to a data format that can be ingest by the models. Moreover, the data integration system's response time needs to be relatively low to forecast and monitor time-sensitive weather events like hurricanes, storms, and floods which require rapid and frequent execution of NWMs.

Weather data processed by one \acrshort{nwm} maybe used by several other \acrshort{nwm}s, as well as by many third-party applications and researcher. For example, logistic companies could use the processed data with their models to plan and schedule their deliveries. Agricultural insurance companies can warn the farmers, as well as calculate premiums based on the anticipated weather patterns. Besides, farmers could rely on weather forecasts to plan their work. % decide what work to do on a given day. For example, drying hay is only possible in dry weather. Also, the aeronautical industry is particularly sensitive to weather conditions, precise weather forecasts are essential since before takeoff a flight need to check the conditions on expected route and at destination.
%required a provide platform to provide such 3rd party access, and use the data or process those based on their applications.

Data Integration and Analysis System (DIAS) \cite{Kawasaki2018DataReduction} and Meteorological Assimilation Data Ingest System (MADIS) \cite{Macdermaid2005ArchitectureP2.39} are two of the well-known Weather Data Integration and Assimilation (WDIA) systems. However, they are proprietary. Further, while \acrfull{fews} \cite{Werner2013TheSystem} is free to use, it is not open source. While a few other data integration systems exists, most of them are also proprietary or closed source. Thus, it is difficult and costly to customize such systems for an island like Sri Lanka which experience different kinds of weather seasons over the year \cite{NaveendrakumarFiveLanka}. % than that software originated need to be highly customized. Throughout the operations and maintenance, it is required to get support with paying lots of charges, and difficult to debug into a closed source system without maintenance support from the vendors.
For example, \acrshort{fews} does not support FLO2D model integration, which is used by \acrshort{curw}'s workflows for water-level forecasting. %In such a case, it required to get support from the vendor to integrate and configure the system.

Existing WDIA systems are based on 
%Most of the software existing out there is not up to date with the latest technology concepts such as Cloud Computing. The modern architecture pattern such as microservice architecture was able to provide high scalability and availability, rather using 
the monolithic or client service architecture; hence, cannot benefit from technologies such as cloud computing to achieve high scalability and availability. Moreover, such systems cannot be updated or reconfigured without taking the system off line. Furthermore, they are platform specific. For example, while \acrshort{fews} can run on either Windows or Linux, downtime is needed to introduce any changes to the system. 
% Most existing systems required to shut down to update the system configuration such as \acrshort{fews}, but a system that is accessed by any third parties should be operating 99.9\% without any downtime and should allow adding new features on the fly with rollover capabilities. The proposed system should be independent of the platforms. But most software out there is platform dependent, as an example \acrshort{fews} can run on Windows and Linux. And should be independent underline infrastructure providers which gives much flexibility for the users to decide the best cost-effective way to host it. 

Given the above reasons, there is a necessity to develop a WDIA system that could handle large volumes of multimodel data efficiently while providing scalability and high throughput. Moreover, it is desirable to support controlled data sharing with other NWMs and third-party access to enable multiple use cases. Furthermore, such a system should be architectured to benefit from technologies such as cloud computing and microservices.
