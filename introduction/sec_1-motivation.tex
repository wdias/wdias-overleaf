Weather prediction can use to reduce the lost in natural disasters and in order to gain more advantage over the natural resources like water management. The weather predictions are performed via \acrfull{nwm} by providing different configurations and the current status of the environment such as temperature, rainfall, and wind direction etc. For weather prediction \acrfull{wrf} model is used for predicting the rain fall, temperature, and wind direction etc. Then using the real time observations, the correct model is selected and after removing the bias, extract the information and feed into the next models. There are lots of \acrshort{nwm} available there, but another example can be taken as FLO2D which can be used to predict the water level on the ground. For the corresponding model developed based on the geo data, giving the current status of water levels and forecasted rainfall, the model predict the forecasted water level.

CUrW SL is a government ministry which is working on weather data forecasting for Sri Lanka in order to reduce the natural disaster effects in Sri Lanka. In order to do that, it is implementing its own system with the idea of long term maintainability rather depend on existing vendors. It collected real time weather data from weather stations and other government entities, and do the forecasting daily. Then the generated results are shared to the thrusted parties. One of the future goals of CUrW SL to open up to the public to share the data, and use of the processed data via their system for the betterment of the society. The institute is operating via the paid TAX of Sri Lankan, and its trying to improve the quality of the life and the economic losts which are effected on the country via sharing the data for external parties to use of them.

%%%%%%%%%%%%%%%%%%%%%%%%%%%%%%%%%%%%%%%%%%%%%%%%%%%%%%%%%%%%%%%%%%%%%%%%%%%%%%%%
\section{Motivation}
To enhance the accuracy of weather predictions, it is necessary to provide reliable and detailed weather data as inputs to \acrfull{nwm}. These NWMs utilize \db{multimodal} weather data collected via diverse sources such as automated weather stations, radars, air balloons, and satellite images. Prior to feeding such diverse data (collected from different sources that belongs to different stakeholders) into respective NWMs, it is necessary to integrate data into a common format. Moreover, the data integration system’s response time need to be relatively low to forecast and monitor time-sensitive weather events like hurricanes, storms, and floods which require rapid and frequent execution of NWMs.

Processed weather data needs to be accessed by many third-party applications and research in order to make use of them. For example, logistic companies could use those data with their models to plan and schedule their deliveries. Agricultural insurance companies can warn the farmers in advance, as well as calculate premiums based on anticipated weather patterns. Also farmers rely on weather forecasts to decide what work to do on any particular day. For example, drying hay is only feasible in dry weather. Furhter the aviation industry is especially sensitive to the weather, accurate weather forecasting is essential, since prior to takeoff on the conditions to expect en route and at their destination. 
Thus it is required a platform to provide such 3rd party access, and use the data or process those based on their applications.

While Data Integration and Analysis System (DIAS) and Meteorological Assimilation Data Ingest System (MADIS) are some of well-known Weather Data Integration and Assimilation (WDIA) systems, they are proprietary. Further Delft-FEWS is free to use, it is not open source. Hence, users of WDIA are forced to pay heavy licenses and are unable to extend the solutions to cater to their country-specific requirements. Therefore, the objective of this research is to develop a WDIA system for Sri Lanka, as well as make it open source, so that others could user and contribute to the solution.

Even though there are many data integration systems, most of them are proprietary or close sourced. Which means for an island like Sri Lanka that having different kinds of weather seasons over the year than those software originated need to be highly customized. Throughout the operations and maintenance, it is required to get support with paying lots of charges, and difficult to debug into a closed source system without maintenance support from the vendors. As an example, \acrshort{fews} does not support FLO2D model integration, and CUrW SL is using FLO2D for water level forecasting in the data flow. In such a cases, it required to get the support from the vendor in order to integrate and configure the system.
And most of the software existing out there is not up to date with the latest technology concepts such as Cloud Computing. \acrfull{wdias} uses the modern architecture pattern such as microservice in order to achieve scalability, rather using monolithic architecture or client service architecture. Most existing systems required to shutdown in order to update the system configuration such as \acrshort{fews}, but a systems which is access by many third parties should be operate 99.9\% without any downtime and should allows to add new features on the fly with rollover capabilities. The proposed system should be independent of the platform. But most software out there is not, as example \acrshort{fews} can run on Windows and Linux. And should be independent underline infrastructure providers which give much flexibility for the users to decide the best cost effective way to host it. Given the above facts, there is a necessary for such a systems to be implemented.
