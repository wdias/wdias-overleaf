\section{Problem Statement}
Handling spatio-temporal weather data is challenging, as it involves large data volumes requiring high computing, storage, and network resources. Moreover, the data integration system needs to be capable of interacting with different data formats including spatio-temporal bulk stream data. Furthermore, it should support geographical and time-based queries, as well as provide third-party access to data with easy integration. Scalability and high availability are also essential to support different use cases and mission-critical nature of applications.
Further, users should be able to easily deploy and manage the system while benefiting from technological advances. %with modern technologies and should be able to get the advantage of technology advancement. 
In this context, the research problem can be stated as follows:
 
\emph{How to design an extendable weather data integration, assimilation, and dissemination system that is capable of handling large volumes of multimodel weather data while providing high throughput, scalability, and availability?}

\section{Objectives}
This research addresses the above problem statement by achieving the following research objectives:
\begin{itemize}
    \item To develop an open platform to integrate weather data from different sources in different formats
    \item To design and develop a schema to store multidimensional weather timeseries data while optimizing access time, availability, and storage
    \item To optimize schema to support data queries based on geography and time
    \item To develop an API for third-party data sharing
%    \item To use open source tools and provide the source publicly with appropriate liecense.
\end{itemize}
