\section{Problem Statement}
Handling Spatio-temporal weather data is challenging \dbc{as} it involves large data volumes required to have a higher number of resources such as computation, storage, and network bandwidth. Moreover, the data integration system needs to be capable of interacting with different data formats including Spatio-temporal bulk stream data. Furthermore, it should support geographical and time-based queries, as well as provide third-party access to data with easy integration. Scalability and high availability are also essential to support different use cases and mission-critical nature of applications.
And users should be able to easily deploy and manage the system with modern technologies and should be able to get the advantage of technology advancement. In this context, the research problem can be stated as follows:
 
\emph{Design and implement an extendable weather data integration, assimilation, and dissemination system which is capable of handling bulk data efficiently with providing scalability and high throughput with using open source tools and modern cloud computing technologies.}
\dbc{Give a proper problem statement.}
\gkc{UPDATED. Looked into the OneNote samples. I  should follow one of them from the beginning. :( }

\section{Objectives}
We address the above problem statement by achieving the following research objectives:
\begin{itemize}
    \item To develop an open platform to integrate weather data from different sources in different formats
    \item To design and develop a schema to store multidimensional weather timeseries data while optimizing access time, availability, and storage
    \item To optimize schema to support data queries based on geography and time
    \item To develop an API for third-party data sharing
%    \item To use open source tools and provide the source publicly with appropriate liecense.
\end{itemize}
