\documentclass[a4paper,oneside,12pt]{report}
\usepackage{graphicx,mathptmx}
\usepackage{geometry}
 \geometry{
 a4paper,
 %total={160mm,227mm},
 left=40mm,
 top=25mm,
 bottom=40mm, 
 right=25mm,
}
\renewcommand{\baselinestretch}{1.5} 

\usepackage{setspace}
\usepackage{tabu}
\usepackage[acronym,nopostdot,nonumberlist]{glossaries}
\makeglossaries
\usepackage{import}
\usepackage{listings}
\usepackage{xcolor}
\usepackage{dirtytalk}
\usepackage{tabulary}
\usepackage{multirow}
\usepackage[style=ieee]{biblatex}
\usepackage[titletoc]{appendix}
\usepackage{siunitx}
\usepackage[misc,geometry]{ifsym} % https://tex.stackexchange.com/a/107112/181545, http://tug.ctan.org/info/symbols/comprehensive/symbols-a4.pdf

\usepackage{color,soul}
\usepackage{todonotes}
\newcommand{\db}[1]{\textcolor{blue!40}{#1}}
\newcommand{\dbc}[1]{\todo[author=Dilum, inline, color=blue!40]{#1}}
\newcommand{\gkc}[1]{\todo[author=Gihan, inline, color=green!50]{#1}}
\newcommand{\gc}[1]{\todo[author=Gihan Revisit, inline, color=orange!50]{#1}}

% \usepackage{hyperref} %%% \hl{} on sections will fail this
\usepackage[noabbrev,capitalise]{cleveref}

\addbibresource{mendeley.bib}

\definecolor{codegreen}{rgb}{0,0.6,0}
\definecolor{backcolour}{rgb}{0.95,0.95,0.92}
\lstdefinestyle{code-style}{
  backgroundcolor=\color{backcolour}, commentstyle=\color{codegreen},
  basicstyle=\linespread{1.0}\ttfamily\footnotesize,
  breakatwhitespace=false,
  breaklines=true,
  keepspaces=true,
  numbers=left,
  numbersep=5pt,
  showspaces=false,
  showstringspaces=false,
  showtabs=false,
  tabsize=2,
  framesep=0pt
}
\lstset{style=code-style}

\usepackage{newfloat} % Caption Lists: https://tex.stackexchange.com/a/239114/181545
\DeclareFloatingEnvironment[placement={!ht},name=List]{customlist}

\begin{document}
\begin{titlepage}
    \begin{center}
        \Large{
        \textbf{WEATHER DATA INTEGRATION AND ASSIMILATION SYSTEM}}\\
        \vspace{144pt}
  \large      
        %\vspace{1cm}
      % by
       
        
        %\vspace{1.0cm}
        
        Gihan Chanuka Karunarathne\\
        \vspace{24pt}
       % \normalsize
      178004U\\
         \vspace{72pt}
        %\normalsize
        Degree of Master of Science\\
       
        %\centering
         %\includegraphics[width=0.4\textwidth]{0.Title_Page/uom.png}\\
            %\vspace{0.8cm}
         %\normalsize
        
        %\vfill
       \vspace{72pt}
        \large
        Department of Computer Science and Engineering\\
        \vspace{24pt}
        University of Moratuwa\\
        Sri Lanka\\
        \vspace{32pt}
        January 2021
        
    \end{center}
\end{titlepage}

\begin{titlepage}
    \begin{center}
       % \vspace*{1cm}
        \Large{
        \textbf{WEATHER DATA INTEGRATION AND ASSIMILATION SYSTEM}}\\
        \vspace{144pt}
  \large      

        Herath Mudiyanselage Gihan Chanuka Karunarathne\\
        \vspace{24pt}
       % \normalsize
        178004U\\
         \vspace{72pt}
        \normalsize
        Thesis submitted in partial fulfillment of the requirements for the degree Master of Science in Computer Science and Engineering\\
     
        %\vfill
       \vspace{72pt}
        \large
        Department of Computer Science and Engineering\\
        \vspace{24pt}
        University of Moratuwa\\
        Sri Lanka\\
        \vspace{32pt}
        January 2021
        
    \end{center}
\end{titlepage}

\pagenumbering{roman}

\newgeometry{left=4cm,top=0.5cm,right=2.5cm}
\chapter*{Declaration}
\addcontentsline{toc}{chapter}{Declaration}

I declare that this is my own work and this thesis does not
incorporate without acknowledgement any material previously submitted for a
Degree or Diploma in any other University or institute of higher learning and to
the best of my knowledge and belief it does not contain any material previously
published or written by another person except where the acknowledgement is
made in the text.

Also, I hereby grant to University of Moratuwa the non-exclusive right to
reproduce and distribute my thesis/dissertation, in whole or in part in print,
electronic or other medium. I retain the right to use this content in whole or part
in future works (such as articles or books).

\vspace{0.5in}
\noindent
\begin{tabu} to 1.0\textwidth { X[6.5,l] X[3.5,l] }
  Signature: & Date:
\end{tabu}



\vspace{0.5in}
\noindent
The above candidate has carried out research for the Masters thesis under our supervision.


\vspace{0.5in}
\noindent
\begin{tabu} to 1.0\textwidth { X[6.5,l] X[3.5,l] }
  Name of the supervisor: & Dr. HMN Dilum Bandara\\ [2.5ex]
  Signature of the supervisor: & Date:
\end{tabu}

\restoregeometry
\normalsize

\newgeometry{left=2.5cm,top=0.2cm,right=2cm}
\addcontentsline{toc}{chapter}{Abstract} 
{\setstretch{1.5} 

\chapter*{Abstract}

Numerical Weather Models (NWMs) utilize data collected via diverse sources such as automated weather stations, radars, air balloons, and satellite images. Before using such multimodal data in a \acrshort{nwm}, it is necessary to transcode data into a format ingested by the \acrshort{nwm}. Moreover, the data integration system's response time needs to be relatively low to forecast and monitor time-sensitive weather events like hurricanes, storms, and flash floods that require rapid and frequent execution of \acrshort{nwm}s. The resulting weather data also need to be accessed by many researchers and third-party applications such as logistic and agricultural insurance firms.
Existing weather data integration systems are based on monolithic or client-server architectures; hence, unable to benefit from novel computational models such as cloud computing and containerized applications. Moreover, most of these softwares are proprietary or closed-source, making it difficult to customize them for an island like Sri Lanka with different weather seasons.
Therefore, in this research, we propose \acrfull{wdias} that utilizes microservices to achieve scalability, high availability, and low-cost operation based on cloud computing. The use of stateless microservices also enables \acrshort{wdias} to add new features on the fly with rollover capabilities. Moreover, \acrshort{wdias} provides a modular framework to integrate data from different sources, export into different formats, and add new functionality by adding extension modules.
We demonstrate the utility of \acrshort{wdias} using a cloud-based experimental setup and weather-related synthetic workloads.

\vspace{4mm}

\textbf{Keywords:} Cloud computing, data assimilation, data integration, microservice, weather
}

\restoregeometry
\normalsize

\addcontentsline{toc}{chapter}{Dedication} 
\chapter*{Dedication}
I dedicate this thesis work to teachers, lectures, my family and specially colleagues at \acrfull{curw}. A special feeling of gratitude to my loving parents, Nandawathi Dissanayake and H.M.K. Karunarathne whose put countless sweats to support me throughout my entire life. My wife Jayani Kumarasinghe and my son Sasmitha Karunarathne who missed a lots of wonderful moments to give me freedom to work on this research.

Nevertheless I could not forget all of my friends at \acrshort{curw} who have supported and be with me during this period of time. Also, special thanks to Dr. Dilum Bandara and Prof. Srikantha Herath for giving me this wonderful opportunity to explore this new domain.

\addcontentsline{toc}{chapter}{Acknowledgements} 
\chapter*{Acknowledgements}
I would like to thank the members of my evaluation committee who were more than generous with their expertise and their precious time. Special thanks to Dr Dilum Bandara, my research supervisor for helping me with his precious time, reading, encouragement and been patient with pushing me to the next level. Also, I am, thankful to Dr. Dilika Peris, and Dr. Indika Perera for serve as my evaluation panel without any hesitation and Dr. Srikantha Herath for giving more knowledge on weather domain and support as my external supervisor.

I would like to thank the Department of Computer Science and Engineering at the University of Moratuwa for allowing me to conduct my research and provide all the assistance requested. Special thanks go to the academic and non-academic staff of the department for their continued support. I also thank the \acrfull{curw} for providing the domain expertise, access to resources, and financial support during the research.

Finally, I would like to thank the lectures, the evaluation panel and the colleagues who helped me with this project. Their enthusiasm and willingness to provide feedback support me to go far beyond my limits and complete my research with an enjoyable experience.


%%%%%%%%%%%%%%%%%%%%%%%%%%%%%%%%%%%%%%%%%%%%%%%%%%%%%%%%%%%%%%%%%%%%%%%%%%%%%%%%
\tableofcontents

\addcontentsline{toc}{chapter}{List of Figures}
\listoffigures

\addcontentsline{toc}{chapter}{List of Tables} 
\listoftables

%%%%%%%%%%%%%%%%%%%%%%%%%%%%%%%%%%%%%%%%%%%%%%%%%%%%%%%%%%%%%%%%%%%%%%%%%%%%%%%%
\newglossarystyle{custom-glossary-style}{%
  \setglossarystyle{long}%
  \renewcommand{\glsgroupskip}{}
  \renewenvironment{theglossary}%
    {\begin{longtable}[l]{@{}p{\dimexpr 3cm-\tabcolsep}p{0.8\hsize}}}% <-- change the value here
    {\end{longtable}}%
}

\addcontentsline{toc}{chapter}{List of Abbreviations} 
\printglossary[style=custom-glossary-style, type=\acronymtype, title=List of Abbreviations, toctitle=List of Abbreviations]

%%%%%%%%%%%%%%%%%%%%%%%%%%%%%%%%%%%%%%%%%%%%%%%%%%%%%%%%%%%%%%%%%%%%%%%%%%%%%%%%
\chapter{Introduction}
\label{ch:intro}
\pagenumbering{arabic}

\import{introduction/}{sec_1-motivation.tex}

\import{introduction/}{sec_2-problem_and_objectives.tex}

\import{introduction/}{sec_3-outline.tex}

%%%%%%%%%%%%%%%%%%%%%%%%%%%%%%%%%%%%%%%%%%%%%%%%%%%%%%%%%%%%%%%%%%%%%%%%%%%%%%%%
\chapter{Literature Review}
\label{ch:literature}

\import{lit/}{sec_1-fews.tex}

\import{lit/}{sec_2-lead.tex}

\import{lit/}{sec_3-other-systems.tex}

%%%%%%%%%%%%%%%%%%%%%%%%%%%%%%%%%%%%%%%%%%%%%%%%%%%%%%%%%%%%%%%%%%%%%%%%%%%%%%%%
\chapter{Research Methodology}
\label{ch:method}

\import{method/}{sec_0_introduction.tex}

\import{method/}{sec_1-architectural_decisions.tex}

\import{method/}{sec_3-microservice.tex}

\import{method/}{sec_4-db_struct.tex}

\import{method/}{sec_5-data_preprocess.tex}

\import{method/}{sec_6-query.tex}

\import{method/}{sec_7-summary.tex}

%%%%%%%%%%%%%%%%%%%%%%%%%%%%%%%%%%%%%%%%%%%%%%%%%%%%%%%%%%%%%%%%%%%%%%%%%%%%%%%%
\chapter{Performance Analysis}
\label{ch:results}

\import{results/}{sec_0-introduction.tex}

\import{results/}{sec_1-test_plan.tex}

\import{results/}{sec_2-workload.tex}

\import{results/}{sec_3-observations.tex}

\import{results/}{sec_4-discussion.tex}

%%%%%%%%%%%%%%%%%%%%%%%%%%%%%%%%%%%%%%%%%%%%%%%%%%%%%%%%%%%%%%%%%%%%%%%%%%%%%%%%
\chapter{Summary}
\label{ch:summary}

\import{summary/}{sec_1-conclusions.tex}

\import{summary/}{sec_2-research_limitations.tex}

%%%%%%%%%%%%%%%%%%%%%%%%%%%%%%%%%%%%%%%%%%%%%%%%%%%%%%%%%%%%%%%%%%%%%%%%%%%%%%%%

\graphicspath{ {./images/} }
\newacronym{wdias}{WDIAS}{Weather Data Integration and Assimilation System}

\newacronym{fews}{Delft-FEWS} {Delft-FEWS, Deltares}
\newacronym{lead}{LEAD}{Linked Environments for Atmospheric Discovery}
\newacronym{dias}{DIAS}{Data Integration and Assimilation System}
\newacronym{madis}{MADIS}{Meteorological Assimilation Data Ingest System}

\newacronym{nwm}{NWMs}{Numerical Weather Models}
\newacronym{NetCDF}{NetCDF}{Network Common Data Form}
\newacronym{soa}{SOA}{Service Oriented Architecture}
\newacronym{wrf}{WRf}{Weather Research and Forecast}
\newacronym{esb}{ESB}{Enterprise Service Bus}
\newacronym{microservice}{Microservice}{Microservice Architectire}

\newacronym{curw}{CUrW SL}{Urban Center for Water, Sri Lanka}
% \bibliographystyle{plain}
% \addbibresource{mendeley.bib}

\addcontentsline{toc}{chapter}{References}
\renewcommand*{\UrlFont}{\rmfamily}
\printbibliography[title={References}]

%%%%%%%%%%%%%%%%%%%%%%%%%%%%%%%%%%%%%%%%%%%%%%%%%%%%%%%%%%%%%%%%%%%%%%%%%%%%%%%%
\begin{appendices}
    \chapter{WDIAS REST API}
    
\emph{Timeseries Metadata} endpoints allow to register new timeseries into the \acrshort{wdias} system before upload data.
\begin{itemize}
    \item parameter
    \begin{itemize}
        \item \texttt{GET /parameter} -- Get parameters
        \item \texttt{POST /parameter} -- Create parameter
        \item \texttt{PUT /parameter/id} -- Update parameter
        \item \texttt{DELETE /parameter/id} -- Delete parameter
    \end{itemize}
    \item timeStep
    \begin{itemize}
        \item \texttt{GET /timestep} -- Get TimeSteps
        \item \texttt{POST /timestep} -- Create TimeStep
        \item \texttt{PUT /timestep/id} -- Update TimeStep
        \item \texttt{DELETE /timestep/id} -- Delete TimeStep
    \end{itemize}
    \item location
    \begin{itemize}
        \item \texttt{GET /location} -- Get location points
        \item \texttt{POST /location} -- Create location point
        \item \texttt{PUT /location/id} -- Update location point
        \item \texttt{DELETE /location/id} -- Delete location point
        \item \texttt{GET /location/regular-grid} -- Get regular grids
        \item \texttt{POST /location/regular-grid} -- Create regular grid
        \item \texttt{PUT /location/regular-grid/id} -- Update regular grid
        \item \texttt{DELETE /location/regular-grid/id} -- Delete regular grid
    \end{itemize}
    \item timeseries
    \begin{itemize}
        \item \texttt{GET /timeseries} -- Get timeseries
        \item \texttt{POST /timeseries} -- Create timeseries
        \item \texttt{PUT /timeseries/id} -- Update timeseries
        \item \texttt{DELETE /timeseries/id} -- Delete timeseries
    \end{itemize}
\end{itemize}

\emph{Import} endpoints allow to upload data into the \acrshort{wdias} system.
\begin{itemize}
    \item \texttt{POST /import/json/raw} -- Insert Scalar/Vector timeseries data with JSON body
    \item \texttt{POST /import/csv/raw} -- Insert Scalar/Vector timeseries data in CSV format
    \item \texttt{POST /import/ascii-grid/upload} -- Upload set of ASCII Grid files
    \item \texttt{POST /import/ascii-grid/binary} -- Upload a single ASCII Grid file
\end{itemize}

\emph{Export} endpoints allow to get/download data from the \acrshort{wdias} system.
\begin{itemize}
    \item \texttt{GET /export/json/raw} -- Get Scalar/Vector timeseries data in JSON format
    \item \texttt{GET /export/json/raw?start=2000-01-01\&end=2000-01-03} -- Get Scalar/Vector timeseries data in JSON format for given time period
    \item \texttt{GET /export/netcdf/binary} -- Download Grid data timeseries data in \acrshort{netCDF} format
\end{itemize}
    \chapter{WDIAS Query API}
    \addtocontents{toc}{\protect\setcounter{tocdepth}{0}} 
\chapter{WDIAS Query API}

\subsection{Timeseries Metadata Queries}
Get timeseries metadata by timeseriesID.
\begin{lstlisting}
    GET <HOST_ADDR>/metadata/timeseries/<TIMESERIES_ID>
\end{lstlisting}

Search for timeseries metadata with key attributes of the timeseries.
\begin{lstlisting}
    GET <HOST_ADDR>/metadata/timeseries?<QUERY_STRING}
\end{lstlisting}

QUERY\_STRING can be consist of zero to multiple parameters. Multiple query parameters are separated by the ampersand, "\&".
\begin{itemize}
    \item \texttt{moduleId} (Optional), e.g., HEC-HMS
    \item \texttt{valueType} (Optional), e.g., Scalar
    \item \texttt{parameterId} (Optional), e.g., O.Precipitation
    \item \texttt{locationId} (Optional), e.g., wdias\_hanwella
    \item \texttt{timeseriesType} (Optional), e.g., ExternalHistorical
    \item \texttt{timeStepId} (Optional), e.g., each\_15\_min
\end{itemize}
For the moment, query are only support by strict matching with above parameter values.

%%%%%%%%%%%%%%%%%%%%%%%%%%%%%%%%%%%%%
\subsection{Timeseries Geo Queries}

\paragraph{Query Locations within Area}-- Following endpoint supports to query the locations which exist within the area provided by the GeoJson data in the JSON payload.
\begin{lstlisting}
    POST <HOST_ADDR>/query/location
    JSON Body:
    {
        "geoJson": {
            "type": "Polygon",
            "coordinates": [
                [
                    [
                        <longitude>,
                        <latitude>
                    ],
                    ...
                ]
            ]
        }
    }
\end{lstlisting}

paragraph{Query parameters in Locations}-- Following endpoint supports to query the parameters stored against given locations. Location IDs should be provided as a list of JSON payload.payload.
\begin{lstlisting}
    POST <HOST_ADDR>/query/parameter
    JSON Body:
    ["<locationID>", ...]
\end{lstlisting}

\paragraph{Query Timeseries by Location}-- Following endpoint supports to query the timeseries metadata for a given location.
\begin{lstlisting}
    POST <HOST_ADDR>/query/timeseries
    JSON Body:
    {
        "location": "<LOCATION_ID>"
    }
\end{lstlisting}

\subsubsection{Query Timeseries by Locations}
Following endpoint supports to query the timeseries metadata for a given location set. Location IDs should be provided as a list in the locations field.
\begin{lstlisting}
    POST <HOST_ADDR>/query/timeseries
    JSON Body:
    {
        "locations": ["LOCATION_ID", ...]
    }
\end{lstlisting}

\subsubsection{Query Timeseries by Parameter}
Following endpoint supports to query the timeseries metadata by parameter for given location/locations.
\begin{lstlisting}
    POST <HOST_ADDR>/query/timeseries
    JSON Body:
    {
        "locations": [""],
        "parameter": ""
    }
\end{lstlisting}

\paragraph{Query Timeseries in Area} -- Following endpoint supports to query the timeseries metadata within a given area (specified as a polygon in geoJson format).
\begin{lstlisting}
    POST <HOST_ADDR>/query/timeseries
    JSON Body:
    {
        "geoJson": {
            "type": "Polygon",
            "coordinates": [
                [
                    [
                        <longitude>,
                        <latitude>
                    ],
                    ...
                ]
            ]
        }
    }
\end{lstlisting}

\paragraph{Query Timeseries in Area by Parameter}-- Following endpoint supports to query the timeseries metadata within a given area for a given parameter. This endpoint is an extension of the above endpoint.

\begin{minipage}{\linewidth}
\begin{lstlisting}
    POST <HOST_ADDR>/query/timeseries
    JSON Body:
    {
        "geoJson": {
            "type": "Polygon",
            "coordinates": [
                [
                    [
                        <longitude>,
                        <latitude>
                    ],
                    ...
                ]
            ]
        },
        "parameter": "<PARAMETER_ID>"
    }
\end{lstlisting}
\end{minipage}

\paragraph{Query All Timeseries} -- Send a request with an empty JSON body that will respond with all the timeseries.
\begin{lstlisting}
    POST <HOST_ADDR>/query/timeseries
    JSON Body:
    {}
\end{lstlisting}

\end{appendices}

% \appendix
% \addcontentsline{toc}{section}{Appendix~\ref{apx:rest_api}: WDIAS REST API}
% \chapter{WDIAS REST API}
% 
\emph{Timeseries Metadata} endpoints allow to register new timeseries into the \acrshort{wdias} system before upload data.
\begin{itemize}
    \item parameter
    \begin{itemize}
        \item \texttt{GET /parameter} -- Get parameters
        \item \texttt{POST /parameter} -- Create parameter
        \item \texttt{PUT /parameter/id} -- Update parameter
        \item \texttt{DELETE /parameter/id} -- Delete parameter
    \end{itemize}
    \item timeStep
    \begin{itemize}
        \item \texttt{GET /timestep} -- Get TimeSteps
        \item \texttt{POST /timestep} -- Create TimeStep
        \item \texttt{PUT /timestep/id} -- Update TimeStep
        \item \texttt{DELETE /timestep/id} -- Delete TimeStep
    \end{itemize}
    \item location
    \begin{itemize}
        \item \texttt{GET /location} -- Get location points
        \item \texttt{POST /location} -- Create location point
        \item \texttt{PUT /location/id} -- Update location point
        \item \texttt{DELETE /location/id} -- Delete location point
        \item \texttt{GET /location/regular-grid} -- Get regular grids
        \item \texttt{POST /location/regular-grid} -- Create regular grid
        \item \texttt{PUT /location/regular-grid/id} -- Update regular grid
        \item \texttt{DELETE /location/regular-grid/id} -- Delete regular grid
    \end{itemize}
    \item timeseries
    \begin{itemize}
        \item \texttt{GET /timeseries} -- Get timeseries
        \item \texttt{POST /timeseries} -- Create timeseries
        \item \texttt{PUT /timeseries/id} -- Update timeseries
        \item \texttt{DELETE /timeseries/id} -- Delete timeseries
    \end{itemize}
\end{itemize}

\emph{Import} endpoints allow to upload data into the \acrshort{wdias} system.
\begin{itemize}
    \item \texttt{POST /import/json/raw} -- Insert Scalar/Vector timeseries data with JSON body
    \item \texttt{POST /import/csv/raw} -- Insert Scalar/Vector timeseries data in CSV format
    \item \texttt{POST /import/ascii-grid/upload} -- Upload set of ASCII Grid files
    \item \texttt{POST /import/ascii-grid/binary} -- Upload a single ASCII Grid file
\end{itemize}

\emph{Export} endpoints allow to get/download data from the \acrshort{wdias} system.
\begin{itemize}
    \item \texttt{GET /export/json/raw} -- Get Scalar/Vector timeseries data in JSON format
    \item \texttt{GET /export/json/raw?start=2000-01-01\&end=2000-01-03} -- Get Scalar/Vector timeseries data in JSON format for given time period
    \item \texttt{GET /export/netcdf/binary} -- Download Grid data timeseries data in \acrshort{netCDF} format
\end{itemize}
% \addtocontents{toc}{\protect\setcounter{tocdepth}{0}} % Issue with List of figures
% \chapter{WDIAS Query API}
% \addtocontents{toc}{\protect\setcounter{tocdepth}{0}} 
\chapter{WDIAS Query API}

\subsection{Timeseries Metadata Queries}
Get timeseries metadata by timeseriesID.
\begin{lstlisting}
    GET <HOST_ADDR>/metadata/timeseries/<TIMESERIES_ID>
\end{lstlisting}

Search for timeseries metadata with key attributes of the timeseries.
\begin{lstlisting}
    GET <HOST_ADDR>/metadata/timeseries?<QUERY_STRING}
\end{lstlisting}

QUERY\_STRING can be consist of zero to multiple parameters. Multiple query parameters are separated by the ampersand, "\&".
\begin{itemize}
    \item \texttt{moduleId} (Optional), e.g., HEC-HMS
    \item \texttt{valueType} (Optional), e.g., Scalar
    \item \texttt{parameterId} (Optional), e.g., O.Precipitation
    \item \texttt{locationId} (Optional), e.g., wdias\_hanwella
    \item \texttt{timeseriesType} (Optional), e.g., ExternalHistorical
    \item \texttt{timeStepId} (Optional), e.g., each\_15\_min
\end{itemize}
For the moment, query are only support by strict matching with above parameter values.

%%%%%%%%%%%%%%%%%%%%%%%%%%%%%%%%%%%%%
\subsection{Timeseries Geo Queries}

\paragraph{Query Locations within Area}-- Following endpoint supports to query the locations which exist within the area provided by the GeoJson data in the JSON payload.
\begin{lstlisting}
    POST <HOST_ADDR>/query/location
    JSON Body:
    {
        "geoJson": {
            "type": "Polygon",
            "coordinates": [
                [
                    [
                        <longitude>,
                        <latitude>
                    ],
                    ...
                ]
            ]
        }
    }
\end{lstlisting}

paragraph{Query parameters in Locations}-- Following endpoint supports to query the parameters stored against given locations. Location IDs should be provided as a list of JSON payload.payload.
\begin{lstlisting}
    POST <HOST_ADDR>/query/parameter
    JSON Body:
    ["<locationID>", ...]
\end{lstlisting}

\paragraph{Query Timeseries by Location}-- Following endpoint supports to query the timeseries metadata for a given location.
\begin{lstlisting}
    POST <HOST_ADDR>/query/timeseries
    JSON Body:
    {
        "location": "<LOCATION_ID>"
    }
\end{lstlisting}

\subsubsection{Query Timeseries by Locations}
Following endpoint supports to query the timeseries metadata for a given location set. Location IDs should be provided as a list in the locations field.
\begin{lstlisting}
    POST <HOST_ADDR>/query/timeseries
    JSON Body:
    {
        "locations": ["LOCATION_ID", ...]
    }
\end{lstlisting}

\subsubsection{Query Timeseries by Parameter}
Following endpoint supports to query the timeseries metadata by parameter for given location/locations.
\begin{lstlisting}
    POST <HOST_ADDR>/query/timeseries
    JSON Body:
    {
        "locations": [""],
        "parameter": ""
    }
\end{lstlisting}

\paragraph{Query Timeseries in Area} -- Following endpoint supports to query the timeseries metadata within a given area (specified as a polygon in geoJson format).
\begin{lstlisting}
    POST <HOST_ADDR>/query/timeseries
    JSON Body:
    {
        "geoJson": {
            "type": "Polygon",
            "coordinates": [
                [
                    [
                        <longitude>,
                        <latitude>
                    ],
                    ...
                ]
            ]
        }
    }
\end{lstlisting}

\paragraph{Query Timeseries in Area by Parameter}-- Following endpoint supports to query the timeseries metadata within a given area for a given parameter. This endpoint is an extension of the above endpoint.

\begin{minipage}{\linewidth}
\begin{lstlisting}
    POST <HOST_ADDR>/query/timeseries
    JSON Body:
    {
        "geoJson": {
            "type": "Polygon",
            "coordinates": [
                [
                    [
                        <longitude>,
                        <latitude>
                    ],
                    ...
                ]
            ]
        },
        "parameter": "<PARAMETER_ID>"
    }
\end{lstlisting}
\end{minipage}

\paragraph{Query All Timeseries} -- Send a request with an empty JSON body that will respond with all the timeseries.
\begin{lstlisting}
    POST <HOST_ADDR>/query/timeseries
    JSON Body:
    {}
\end{lstlisting}


\end{document}
