Many weather forecasting, assimilation, and dissemination systems are developed to reduce the damage causing by natural disasters such as floods, storms, hurricanes, and even droughts.
Other countries are also using those systems while adopting them to specific weather conditions.
This chapter presents a literature review on existing systems and their architectural designs. In \cref{se:fews}, we present \acrfull{fews}, an open model integration framework. \acrfull{lead}, an open data handling platform distributed as closed-source software, is presented in \cref{se:lead}.
In \cref{se:dias}, we discuss \acrfull{dias} which is a common data platform that can manage weather data. \acrfull{madis} is another widely used data integration system, which also provides data access with quality controls is presented in \cref{se:madis}. Under each system, we explore their system architecture, scalability, and flexibility.


%%%%%%%%%%%%%%%%%%%%%%%%%%%%%%%%%%%%%%%%%%%%%%%%%%%%%%%%%%%%%%%%%%%%%%%%%%%%%%%%
\section{\acrshort{fews} flow forecasting system}
\label{se:fews}

\acrshort{fews} \cite{Werner2013TheSystem} was developed by Deltares in the Netherland for operational forecasting agencies. The \acrshort{fews} codebase is not fully open source at the moment.
\hl{Most of models are using a model-centric approach such as inputs, need to be in the format of specific to the model}. Also, the outputs produced by the models are specific to the model; hence, difficult to integrate with other systems. For example, FLO2D creates human readable text output files. However, as \acrshort{fews} uses a modular approach it is easier to integrate new models. 
Thus, \acrshort{fews} can be considered as an integration framework or a middleware for the models.

\hl{Forecasting processes is creation of modeling steps with combining data transformation algorithms. In the flow}, each step process and feed data into the next step. The \acrshort{fews} is flexible enough to integrate new models and algorithms into the core codebase that can use to create new workflows \cite{Werner2013TheSystem}. %\acrshort{fews} does not have any inbuilt hydrological modeling capabilities within its codebase. Rather it 
\acrshort{fews} is a framework that can be used to integrate into the system and create workflows for forecasting.

Are proposed by Haggett \cite{Haggett1998AnWales} key elements of a forecasting system are defined as detection, forecasting, dissemination and warning, and responding to the events. The \acrshort{fews} focuses on the forecasting step out from above steps. The main objective of this step is to provide additional lead time by forecasting future hydro-meteorological conditions \cite{Werner2005FloodCatchments}. It is a valid argument that providing accurate predictions with greater lead time can reduce the level of destruction. To forecast, the system should be capable of integrating real-time data from meteorological and hydrological observation network systems, and the disseminate the forecasted results through relevant products to the warning process.

\cref{fi:fews_schematic} shows the connection between the forecasting system to real-time data integration systems and dissemination systems. One of the widely used use cases is to use meteorological forecast data to estimate precipitation, and then use a hydrological and hydraulic model chain to predict the surface water level affects. The hydrological and hydraulic model should be design based on the geographical data. The ground should be analyzed geographically and divide into the catchments. Based on the affected catchment, it can be further divided into sub-catchments to reduce the complexity of the simulation task. Ideally, the forecasting system should be flexible enough to allow modification of models and data, while keeping the most constant way possible of working with forecasters.

\begin{figure}[htp]
    \centering
    \includegraphics[width=0.8\textwidth]{lit/fews/Schematic-structure-of-a-fl-ood-forecasting-system-showing-the-position-of-Delft-FEWS_W640.png}
    \caption[Schematic structure of a flood forecasting system including \acrshort{fews} and communication among other operational systems]{Schematic structure of a flood forecasting system including \acrshort{fews} and communication with other operational systems \cite{Werner2013TheSystem}.}
    \label{fi:fews_schematic}
\end{figure}

The \acrshort{fews} follows a data-centric approach, where it provides a common data-model to interact with other components. All the timeseries data types are stored in a database which follows a common data model. New models are integrate to the system via using a one of the interfaces provided to interact with the common data-model \cite{Werner2013TheSystem}. \hl{As mention in the paper}, having a common data-model makes it easier to store data efficiently. Operations like reporting and sharing the data \hl{also make it easy by integrating models.} %But the problem with 
However, this approach cannot handle multiple data formats. \acrshort{fews} overcome this issue by introducing adapters for most of the commonly used data formats. While this enables users to easily import data into the system, it also introduces adding additional complexity to the system.

The timeseries data is \hl{one of scalar, vector, or grid data, and all of different types are uniformly stored} as binary objects in a time series table inside the \acrshort{fews}. Functional compenents or any integrated models do not have direct access to the timeseries table which mentioned above, and those should use the data access module to access the data \cite{Werner2013TheSystem}. As explained in the system interpretation, all data access requests need to go through the data access module increasing the cost of data access. Because it stores different data types as binary objects, there is a penalty to convert data into binary objects and vice versa. \hl{Storing all the data in a timeseries table causing all the requests to come into a single data model in a database. Even it gives the advantage of accessing all the data in one place.} Further the performance goes down due to the need to scream high volume of data.% cause a heavy load on the system, causing to slow down in the performance of trying to use the system on a large set of data.

Given the above drawback on storing the data in the system, the \acrshort{fews} data model store the timeseries by \hl{uniquely identifying based on the location, data type, and source of the data} \cite{Werner2013TheSystem}. This allows indexing the database using above key fields \hl{and stores the data separately}, to take advantage of \hl{putting multiple data resources into the system}. As an example, instead of storing the data at a single timeseries table, it is possible to separate and index the database based on the source such as an external source of data or hydrological model of which the timeseries is a result. Or separate by data type and store in multiple storages give the capability for the system to scale with a factor of identical types that can be identified. Example of separation by data types of scalar, vector and gridded will increase the scaling factor by 3x.

Data processing and manipulation \hl{is} a required process in weather forecasting. 
The data integrated from external sources are not in the appropriate temporal and spatial format that can directly feed as an input to a forecasting model or use in other application. Therefore, generic data processing steps are required for most models in a forecasting environment. Such examples are serial and spatial interpolation, data validation, aggregation and disaggregation, and data fusion \cite{Werner2013TheSystem}. This a vital feature in a forecasting system, and affect the quality and accuracy of the predicted data outputs. Because of the common data model concept in \acrshort{fews}, data processing via these functions is much effective. The system is \hl{self-provided some of the generic functions for data processing}, but required algorithms can be implement as a new Java class via \hl{communicate} with the \acrfull{api}. %It is obvious that 
For additional feature integration, users should implement the new functional extensions using Java. %Users do not have the flexibility to development with some other languages they are familiar with or use the support from another language. As an example, Python language is easy to use for beginners and many data scientists are using this programming language for development and it has many data available libraries for data processing. However, in \acrshort{fews}, users cannot take advantage of such existing tools.
\dbc{I removed part of where you are trying to say support for Python is good. Rather than pointing out such minor issues (these are special purpose software), in Chap 3 and 5 you can highlight that by using microservices models/functions can be developed by any language of choice.}

\begin{figure}[htp]
    \centering
    \includegraphics[width=1.0\textwidth]{lit/fews/Architecture-of-Delft-FEWS-showing-the-data-base-the-data-access-layers-and-examples-of_W640.png}
    \caption[Architecture of \acrshort{fews}]{Architecture of \acrshort{fews} \cite{Werner2013TheSystem}.}
    \label{fi:fews_data_layer}
\end{figure}

\acrshort{fews} provides data processing and modeling library to access scalar and grid timeseries the same way. However, \hl{the communicates with the database} only via the data access layer as shown in \cref{fi:fews_data_layer} \cite{Werner2013TheSystem}. As %it is shown in the figure, it is clearly showing that 
shown in figure, the system depends on a single database and \hl{via} the data access layer, all the requests are coming to the database through a connection. It is possible to setup and connect to an enterprise-level database with clustering and shading as a paid solution to enhance the throughput. %serve many requests as possible. 
However, the design itself inherently suffers from the database bottleneck which \hl{effected} by running a large set of timeseries or models.

\begin{figure}[htp]
    \centering
    \includegraphics[width=0.9\textwidth]{lit/fews/Linking-Delft-FEWS-with-external-models-The-fi-gure-shows-the-fl-ow-of-data-through-XML_W640.png}
    \caption[\acrshort{fews} integration with external models]{Integrating \acrshort{fews} with external models \cite{Werner2013TheSystem}.}
    \label{fi:fews_general_adapter}
\end{figure}
One of the simple and most effective features in the \acrshort{fews} is an open approach to the integration of models and data. The concept of the open modeling framework is used by the \acrshort{fews} which is proposed by open model integration \cite{Kokkonen2003InterfacingXML}. It simply gives the flexibility to allow operators to integrate more models as well as variations of the same model and come up with new integration flows as much as possible.
The users need to mentioned the input timeseries via XML configurations, and \acrshort{fews} generates the input data as XML files. Then using an adapter \hl{developed for the model transforms the data into required format for the model as a preprocessing step}. Next, \acrshort{fews} executes the model which generates the output. In the post processing step, the results of the model is transformed to an XML file, which \hl{can import} into the data model as shown in \cref{fi:fews_general_adapter} \cite{Werner2013TheSystem}. %The process is simple and consistent throughout all the model integration. 
The model execute by the \acrshort{fews} or the model adapter which is causing tide coupling into the execution process in the system. Users do not have the flexibility to run the models with different configurations such as running with parallel execution. That may be overcome by triggering an external process at the execution time, but it introduces more difficulty for handling after model executed successfully, preceding to the next step in the process. \hl{This part of the system feature} is not focused on the \acrshort{wdias} and user has to come up with own flow or use existing scientific flow management system. Yet, it is a concept that needs to be discussed and understand properly to design a data integration and assimilation system. However, the \acrshort{wdias} is capable of integrating workflow mechanism with its extendable architecture.

Data exchange with the model is mainly done via XML files. \hl{It is possible to happen}, these XML files become very large, which can lead to I/O bottlenecks and causing performance issues \cite{Werner2013TheSystem}. %This issue is discussed in previous paragraphs and 
The authors of the \acrshort{fews} \cite{Werner2013TheSystem} seem to have noticed and accepted this issue. % it here. 
\hl{For overcoming this issue} they introduced file-based exchange of data. This includes the use of binary XML files, the streaming of files via memory and the use of \acrshort{netCDF} files. But far as for the users, it seems to be adding more complexity and context to get higher performance via the system.

The \acrshort{fews} provides a list of adapter available. If an adapter is not available for a given model, the user has to implement the adapter for the model and \acrshort{fews} has the flexibility creating a new adapter. But the effort that required to develop an adapter for a model will vary depending on the complexity of the model required data formats \cite{Werner2013TheSystem}. If the users want to use an existing adapter and need to configure something not supported via the existing adapter, it is hard to get done. Also, there are some cases it is hard to develop an adapter or running alongside the \acrshort{fews}. For example, the FLO2D model used for hydrologic modeling only runs on Windows.\hl{ If you setup the system on Linux based operating system and using Linux based models, it is hard for users to come up with a solution for integrating} the FLO2D models.
The \acrshort{fews} support creating workflow process and allow users to run multiple models parallel or run those sequentially as needed, but the scope of the \acrshort{wdias} is not focused on adding such a feature in the system, rather user has to use a scientific workflow management system or come up with own version of workflow.
