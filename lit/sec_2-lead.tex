\section{Linked Environments for Atmospheric Discovery}
\label{se:lead}

\acrfull{lead} \cite{Droegemeier2005Service-OrientedWeather} addresses the fundamental research challenges needed to create an integrated and scalable framework for adaptive analysis and prediction of the atmosphere. To predict and analyze weather models by researchers, \hl{it required} many resources. Rather than each researcher is running and handling their computer resources to do the weather experiments, \acrshort{lead} is providing pool of resources, then the researchers can use this resource a pool to run their experiment in shorter amounts of time and higher scale. \hl{At the time, researchers are developing their experiment flow, they are not using the resources many, and others are using it at the same time.} The foundation of \acrshort{lead} system is to create a dynamic workflow orchestration and data management in a Web services framework \cite{Droegemeier2005Service-OrientedWeather}.

LEAD's complex range of services, applications, interfaces and IT resources, local and external networks and storage is compiled by users in workflows to study mesoscale incremental manner \cite{Droegemeier2005Service-OrientedWeather}. As it follows the \acrshort{soa}, everything is implemented as independent services. This enables \acrshort{lead} to scale each service as required and update each service without affecting other services. New models that are required for implementing a new forecast flow also has to be implemented as a service and then integrate into the system.

\cref{fi:lead_system} shows the key capabilities of the \acrshort{lead} system. \hl{From the high-level view}, \acrshort{lead} enables users to search and acquire information, simulate and predict weather conditions using digital atmospheric models, assimilate, analyze, use and visualize data and output of models.

\begin{figure}[htp]
    \centering
    \includegraphics[width=1.0\textwidth]{lit/lead/LEAD-system-Fundamental-capabilities-familiar-to-meteorologists-are-shown-in-the-top_W640.png}
    \caption[Layered architecture of LEAD]{Layered architecture of LEAD \cite{Droegemeier2005Service-OrientedWeather}.}
    \label{fi:lead_system}
\end{figure}

The second level contains basic tools such as the following to help to link services together  \cite{Droegemeier2005Service-OrientedWeather}:
\begin{itemize}
    \item the users interact with \acrshort{lead} via a Web portal
    \item data quality control and assimilation happends via ARPS Data Assimilation System (ADAS)
    \item myLEAD is the metadata catalog
    \item \acrfull{wrf} \cite{MesoscaleMicroscaleMeteorologyLaboratoryWeatherModel}
    \item Algorithm Development and Mining is a set of tools for analyze observational data, assimilated data sets, and model output 
    \item Integrated Data Viewer 
\end{itemize}

\hl{The concept of the \acrshort{lead} is a workflow orchestration for on-demand, real-time and dynamically adaptive systems called WOORDS.} The system is doing the workflow orchestration as given in each procedure. Those procedural rules are defined by the researcher and feed into the system. The system is trying to act immediately after the submission, and also it transmits the data, run the models and send back the put with a lower time delay as possible. \hl{According to the load and other requirements, the system will respond to those requirements automatically.}

\begin{figure}[htp]
    \centering
    \includegraphics[width=1.0\textwidth]{lit/lead/LEAD-system-framework-LEAD-is-composed-of-several-interacting-subsystems-with-the-LEAD_W640.png}
    \caption[LEAD system framework]{LEAD system framework \cite{Droegemeier2005Service-OrientedWeather}.}
    \label{fi:lead_framework}
\end{figure}

As shown in \cref{fi:lead_framework}, LEAD consists of the following sub-components, as well as provides a distributed system for integrating and testing LEAD's components:
\begin{itemize}
    \item User subsystem -- comprises the LEAD portal and enable user can access services
    \item Data subsystem -- manages data and metadata, all output of digital models produced by operational or experimental models, and information generated by users.
    \item Tools subsystem -- provides all IT toles and  meteorological tools
    \item Orchestration subsystem -- offers the technologies with which users can manage data flows and model execution flows and can create and own output. It also provides links to other software and processes for continuous or on-demand applications.
\end{itemize}

\begin{figure}[htp]
    \centering
    \includegraphics[width=1\textwidth]{lit/lead/LEADs-service-oriented-architecture-A-wide-variety-of-services-and-resources-grouped_W640.png}
    \caption[LEAD's service-oriented architecture]{LEAD's service-oriented architecture \cite{Droegemeier2005Service-OrientedWeather}.}
    \label{fi:lead_soa}
\end{figure}

%\acrshort{soa}s are widely used in the commercial enterprise sector and form the basis of many scientific "grid" technologies in lead design and development. A variant of \acrshort{soa} was developed later, called microservices architecture, and is now widely used in industry nowadays.

As shown in \cref{fi:lead_soa}, \acrshort{lead} \acrshort{soa} has five different but highly interconnected layers. The bottom layer represents the calculation, application, and raw data sources that are distributed in the LEAD grid and elsewhere. The next level supports web services that provide access to raw services \cite{Droegemeier2005Service-OrientedWeather}. These two layers are working together since \acrshort{lead} system resources are distributed over multiple locations and creating a pool of resources. The upper layer to the raw resources is abstracting the complexity of managing and accessing the resources, and provide simplified access to the upper layer. In the Upper layers, its view as an unlimited resource pool for storing and handling data.

The configuration and execution services of the middle layer, consisting of five elements, represent the services that call LEAD workflows. These are some critical aspects of a weather data management system. Most of the services listed below are required for creating workflows for weather data forecasting. \hl{Thus it} is important to review them and understand the basic needs of a weather data management system.

\begin{itemize}
\item The application-oriented configuration service that manages the implementation and execution of real applications, such as the \acrshort{wrf} simulation model, ADAS and ADaM tools \cite{Droegemeier2005Service-OrientedWeather}. When creating weather workflow, it is required to change the behavior of the model by changing some of the configurations of the model or run a different version of the model.
\item The application resource broker, which matches the appropriate host for execution to each application task, based on the execution’s time constraints \cite{Droegemeier2005Service-OrientedWeather}. This service is a critical part of the system, and responsible for using the resources of the system in an optimized manner. When designing the weather data system, it is required to increase the capacity of the system automatically, and adopt it into the system.
\item The workflow engine service, which manages the experimental workflow instances, invokes both the configuration service and the application resource broker \cite{Droegemeier2005Service-OrientedWeather}. This is a part of workflow orchestration.
\item Catalog services represent how a user or application service discovers data products in the public domain or \acrshort{lead} services \cite{Droegemeier2005Service-OrientedWeather}. This is an important feature that should be available in a weather data system, and users need to search for the availability of the data. \hl{Further wants to analyze} existing data for decision making.
\item Users need a multitude of data services to support comprehensive query, access, and transformation operations on data products. An important goal behind LEAD is transparency of access that facilitates user requests on all available heterogeneous data sources without the negative effects of different formats and naming schemes \cite{Droegemeier2005Service-OrientedWeather}. Users need to have transformation services to read data in different formats. The query service gives the capability to search via available data without any effect on the data formats.
\end{itemize}

Top of \cref{fi:lead_soa} shows the user interface of the \acrshort{lead} which \hl{gives individual services access.} When the user logs into the system, based on the authentication and authorization setting of the account, the portlets can access the services on behalf of the user. %Look into further details is not relevant to developing the \acrshort{wdias} since the focus is not on workflow orchestration.

\acrshort{lead} is a more advanced system which can support mesoscale weather prediction with the effort of multiple universities in the United State America with the effort of many researchers and using many computer resources. At the time it was building, it uses the \acrshort{soa} architecture as into its depth, and with the implementation, each service can scale as needed and possible to enhance the service without interrupting other services. \acrshort{wdias} only focuses on creating a framework that is extendable as system needs more features similar to \acrshort{lead} services.
