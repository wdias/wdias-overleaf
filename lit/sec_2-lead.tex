\section{Linked Environments for Atmospheric Discovery}
\label{se:lead}

\acrfull{lead} addresses the fundamental research challenges needed to create an integrated, scalable framework for adaptive analyzing and predicting the atmosphere \cite{Droegemeier2005Service-OrientedWeather}. To predict and analyze weather models by researchers, it required many resources. Rather than each researcher is running and handling their computer resources to do the weather experiments, \acrshort{lead} is providing pool of resources, then the researchers can use this resource a pool to run their experiment in shorter amounts of time and higher scale. At the time, researchers are developing their experiment flow, they are not using the resources many, and others are using it at the same time. LEAD's foundation is dynamic workflow orchestration and data management in a Web services framework \cite{Droegemeier2005Service-OrientedWeather}.

LEAD's complex array of services, applications, interfaces, and local and remote computing, networking, and storage resources is assembled by users in workflows to study mesoscale weather as it evolves \cite{Droegemeier2005Service-OrientedWeather}. As it follows the \acrshort{soa}, everything is implemented as independent services. This enables \acrshort{lead} to scale each service as required and update each service without affecting other services. New models that are required for implementing a new forecast flow also has to be implemented as a service and then integrate into the system.

\cref{fi:lead_system} shows the fundamental capabilities of the \acrshort{lead} system. From the high-level view, \acrshort{lead} lets users query for and acquire information, simulate and predict the weather by using numerical atmospheric models, assimilate data, analyze, mine, and visualize data and model output.

\begin{figure}[htp]
    \centering
    \includegraphics[width=1.0\textwidth]{lit/lead/LEAD-system-Fundamental-capabilities-familiar-to-meteorologists-are-shown-in-the-top_W640.png}
    \caption[Layered architecture of LEAD]{Layered architecture of LEAD \cite{Droegemeier2005Service-OrientedWeather}.}
    \label{fi:lead_system}
\end{figure}

The second level contains the fundamental tools that help to link services together. This includes the following tools \cite{Droegemeier2005Service-OrientedWeather};
\begin{itemize}
    \item a Web portal, the primary (though not exclusive) user entry point
    \item the ARPS Data Assimilation System (ADAS), a sophisticated tool for data quality control and assimilation
    \item myLEAD, a flexible metadata catalog service
    \item \acrfull{wrf} a next-generation atmospheric prediction and the simulation model
    \item ADaM (Algorithm Development and Mining), a powerful suite of tools for mining observational data, assimilated data sets, and model output 
    \item Integrated Data Viewer (IDV) 
\end{itemize}

The Concept of the \acrshort{lead} is, the workflow orchestration for on-demand, real-time, dynamically adaptive systems, called as WOORDS. The system is doing the workflow orchestration as given in each procedure. Those procedural rules are defined by the researcher and feed into the system. The system is trying to act immediately after the submission, and also it transmits the data, run the models and send back the put with a lower time delay as possible. According to the load and other requirements, the system will respond to those requirements automatically.

\begin{figure}[htp]
    \centering
    \includegraphics[width=1.0\textwidth]{lit/lead/LEAD-system-framework-LEAD-is-composed-of-several-interacting-subsystems-with-the-LEAD_W640.png}
    \caption[LEAD system framework]{LEAD system framework \cite{Droegemeier2005Service-OrientedWeather}.}
    \label{fi:lead_framework}
\end{figure}

As shown in \cref{fi:lead_framework} LEAD consists of the following sub-components, and it provides a distributed testbed for developing, integrating, and testing LEAD's components.
\begin{itemize}
    \item User subsystem -- comprises the LEAD portal and enable user can access services
    \item Data subsystem -- handles data and metadata, any numerical model output produced by operational or experimental models, and user-generated information.
    \item Tools subsystem -- consists of all meteorological and IT tools
    \item Orchestration subsystem -- provides the technologies that let users manage data flows and model execution streams, and create and own output. It also provides linkages to other software and processes for continuous or on-demand applications.
\end{itemize}

\begin{figure}[htp]
    \centering
    \includegraphics[width=1\textwidth]{lit/lead/LEADs-service-oriented-architecture-A-wide-variety-of-services-and-resources-grouped_W640.png}
    \caption[LEAD's service-oriented architecture]{LEAD's service-oriented architecture \cite{Droegemeier2005Service-OrientedWeather}.}
    \label{fi:lead_soa}
\end{figure}

\acrshort{soa}s are widely deployed in the commercial enterprise sector, and they form the foundation of many scientific "grid" technologies at the time of \acrshort{lead} design and developed. A variant of \acrshort{soa} is evolved later called microservice architecture and widely used in the industry nowadays.

As shown in \cref{fi:lead_soa}, \acrshort{lead} \acrshort{soa} has five distinct yet highly interconnected layers. 
The bottom layer represents raw computation, application, and data resources distributed throughout the LEAD grid and elsewhere. The next level upholds the Web services that provide access to raw services \cite{Droegemeier2005Service-OrientedWeather}. These two layers are working together since \acrshort{lead} system resources are distributed over multiple locations and creating a pool of resources. The upper layer to the raw resources is abstracting the complexity of managing and accessing the resources, and provide simplified access to the upper layer. In the Upper layers, its view as an unlimited resource pool for storing and handling data.

The configuration and execution services in the middle layer, consisting of five elements, represent services invoked by LEAD workflows. These are some critical aspects of a weather data management system. Most of the services listed here are required for creating workflows for weather data forecasting. Thus it is important to review them and understand the basic needs of a weather data management system.

\begin{itemize}
\item The application-oriented configuration service that manages the deployment and execution of real applications such as the \acrshort{wrf} simulation model, ADAS, and the ADaM tools \cite{Droegemeier2005Service-OrientedWeather}. When creating weather workflow, it is required to change the behavior of the model by changing some of the configurations of the model or run a different version of the model.
\item The application resource broker, which matches the appropriate host for execution to each application task, based on the execution’s time constraints \cite{Droegemeier2005Service-OrientedWeather}. This service is a critical part of the system, and responsible for using the resources of the system in an optimized manner. When designing the weather data system, it is required to increase the capacity of the system automatically, and adopt it into the system.
\item The workflow engine service, which drives experimental workflow instances, invokes both the configuration service and application resource broker \cite{Droegemeier2005Service-OrientedWeather}. This is a part of workflow orchestration.
\item The catalog services represent how a user or application service discovers public-domain data products or \acrshort{lead} services \cite{Droegemeier2005Service-OrientedWeather}. This is an important feature that should be available in a weather data system, and users need to search for the availability of the data. Further wants to analyze the existing data for decision making.
\item Users require a host of data services to support rich query, access, and transformation operations on data products. An important goal behind LEAD is access transparency facilitating user queries across all available heterogeneous data sources without adverse effects from different formats and naming schemes \cite{Droegemeier2005Service-OrientedWeather}. Users need to have transformation services to read data in different formats. The query service gives the capability to search via available data without any effect on the data formats.
\end{itemize}
The top of \cref{fi:lead_soa}, shows the user interface to the system. This gives access to individual services. When the user logs into the system, based on the authentication and authorization setting bind to the individual account, the portlets can access the services on behalf of the user. Look into further details is not relevant to developing the \acrshort{wdias} since the focus is not on workflow orchestration.

\acrshort{lead} is a more advanced system which can support for mesoscale weather prediction with the effort of multiple universities in the United State America with the effort of many researchers and using many computer resources. At the time it was building, it uses the \acrshort{soa} architecture as into its depth, and with the implementation, each service can scale as needed and possible to enhance the service without interrupting other services. \acrshort{wdias} only focuses on creating a framework that is extendable as system needs more features similar to \acrshort{lead} services.
