\documentclass[conference]{IEEEtran}
\IEEEoverridecommandlockouts
% The preceding line is only needed to identify funding in the first footnote. If that is unneeded, please comment it out.
\usepackage{cite}
\usepackage[acronym]{glossaries}
\usepackage{amsmath,amssymb,amsfonts}
\usepackage{algorithmic}
\usepackage{graphicx}
\usepackage{textcomp}
\usepackage{xcolor}
\def\BibTeX{{\rm B\kern-.05em{\sc i\kern-.025em b}\kern-.08em
    T\kern-.1667em\lower.7ex\hbox{E}\kern-.125emX}}
\begin{document}

\title{Weather Data Integration and Assimilation System\\
% \thanks{Identify applicable funding agency here. If none, delete this.}
}

\author{\IEEEauthorblockN{Gihan Karunarathne}
\IEEEauthorblockA{\textit{dept. computer science and Engineering} \\
\textit{university of moratuwa}\\
Moratuwa, Sri Lanka \\
gihan.09@cse.mrt.ac.lk}
\and
\IEEEauthorblockN{Dr. Dilum Bandara}
\IEEEauthorblockA{\textit{dept. computer science and Engineering} \\
\textit{university of moratuwa}\\
Moratuwa, Sri Lanka \\
dilumb@cse.mrt.ac.lk}
}

\maketitle

%%%%%%%%%%%%%%%%%%%%%%%%%%%%%%%%%%%%%%%%%%%%%%%%%%%%%%%%%%%%%%%%%%%%%%%%%%%%%%%%
\begin{abstract}
We describe our experience with implementing an extendable open source Weather Data Integration and Assimilation System, which is known as \acrshort{wdias}. It uses the modern architecture pattern such as microservice in order to achieve the scalability, rather using monolithic architecture or client service architecture. It provides a modular approach to integration data from different sources and export into different formats. Also the inbuilt extension module system allow users to add new features. The open source tools that are using for \acrshort{wdias} allows to run on Cloud Computing platforms without much hassle with the auto scaling feature allow to run \acrshort{wdias} from 1 CPU node to nodes with few hundred CPUs. The paper describes the initial design goals, evolve over few architectures to get desired performance, and explains how the system in way to achieve scalability. Later we analyse the performance test observations with the performance metrics.
\end{abstract}

\begin{IEEEkeywords}
\acrshort{wdias}, weather, microservice, kubernetes, hierarchical database 
\end{IEEEkeywords}

%%%%%%%%%%%%%%%%%%%%%%%%%%%%%%%%%%%%%%%%%%%%%%%%%%%%%%%%%%%%%%%%%%%%%%%%%%%%%%%%
\section{INTRODUCTION}
I'm going to do this today \cite{Haggett1998AnWales}.

\section{Literature Review}

\subsection{Method}

\subsection{Performance Analysis}

\section{Conclusions}

\section{Summary}

\section*{Acknowledgment}

The preferred spelling of the word ``acknowledgment'' in America is without 
an ``e'' after the ``g''. Avoid the stilted expression ``one of us (R. B. 
G.) thanks $\ldots$''. Instead, try ``R. B. G. thanks$\ldots$''. Put sponsor 
acknowledgments in the unnumbered footnote on the first page.

%%%%%%%%%%%%%%%%%%%%%%%%%%%%%%%%%%%%%%%%%%%%%%%%%%%%%%%%%%%%%%%%%%%%%%%%%
\graphicspath{ {./images/} }
\bibliographystyle{plain}
\bibliography{mendeley}

\end{document}
