\section{Architectural Desicions}
\label{se:architectural_decisions}

Initial design of \acrshort{wdias} used the \acrfull{soa}, and try to use the \acrfull{esb} which can provide capability to integrate different modules, since it is acting as a common layer for all of modules. By default, \acrshort{esb} provides pub sub capabilities.
Since \acrshort{esb} mediator can be used to integrate the modules which are described in \ref{subse:modules_weather_data_integration_sys}. \acrshort{esb} also support different transportation protocols such as HTTP and Web-Socket etc.

But \acrshort{esb} is not a better solution for data streaming or bulk data processing. Also \acrshort{esb} suffer from single point of failure, since all the messages are going though a common bus.

In the second phase of \acrshort{wdias}, the proposed architecture changed to use with Actor model using AKKA framework \cite{HewittWhyDocumentation} after consider into the disadvantage of using \acrshort{esb}. Using the Actors we tried to implement the microservice architecture. In that case, an Actor or An Actor with slave Actors map into each microservice in the \acrshort{wdias}.

The typical \acrshort{soa} model, for example, usually has more dependent \acrshort{esb}s, with microservices using faster messaging mechanisms. \acrshort{soa} also focuses on imperative programming, whereas microservices architecture focuses on a responsive-actor programming style. Moreover, \acrshort{soa} models tend to have an outsized relational database, while microservices frequently use NoSQL or micro-SQL databases (which can be connected to conventional databases). But the real difference has to do with the architecture methods used to arrive at an integrated set of services in the first place.

As mentioned above, \acrshort{esb} support to implement \acrshort{soa}. But \acrshort{soa} has limitations such as single point of failure, slower communication (can’t use for transfer data) etc. So, the microservice architecture get evolved and demanding for the moment. When compare to \acrshort{soa}, microservice architecture has some advantages;
\begin{itemize}
    \item Follow the Single Responsibility Principle
    \item Resilient/Flexible – failure in one service does not impact other services. If you have monolithic or bulky service errors in one service/module it can impact other modules/functionality.
    \item High scalability – demanding services can be deployed in multiple servers to enhance performance and keep away from other services so that they don’t impact other services. Will be difficult to achieve the same with single, large monolithic service.
    \item Easy to enhance – less dependency and easy to change and test
    \item Low impact on other services – being an independent service, this has less chance to impact other services
    \item Easy to understand since they represent the small piece of functionality
    \item Ease of deployment
\end{itemize}

Even with using AKKA framework, it has some of the disadvantage as described in AKKA documentation \cite{Akka.ioWhenCluster}. Even Actor model capable of implementing a microservice architecture, it has few disadvantage. In order to overcome these issues, we moved to the \acrfull{k8s} \cite{LinuxFoundationProduction-GradeKubernetes}.;
\begin{itemize}
    \item Unable to choose technology. But \acrshort{k8s} allows you to choose technology that is best suited for a particular functionality
    \item As mentioned in the AKKA documentation \cite{Akka.ioWhenCluster}, it's better for developing one microservice within the whole \acrshort{wdias} design.
\end{itemize}
