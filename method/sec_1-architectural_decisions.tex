\section{Architectural Decisions}
\label{se:architectural_decisions}

Initial design of \acrshort{wdias} used the \acrfull{soa}. Moreover, it tried to use an \acrfull{esb} to integrate different modules, as it is acting as a common layer for all of modules. By default, \acrshort{esb} provides publisher/subscribe (pub/sub) capabilities.
\hl{Because} \acrshort{esb} \hl{mediator can be used to integrate the modules which are described in Section} \ref{subse:modules_weather_data_integration_sys}. \acrshort{esb} also support different transportation protocols such as HTTP and Web sockets. However, it turned out that \acrshort{esb} is not a suitable for data streaming or bulk data processing. Also, \acrshort{esb} suffer from single point of failure, as all the messages are going though a common bus.

In the second design round of \acrshort{wdias}, we attempted to use the actor model using AKKA framework \cite{HewittWhyModel}. %However, after consider into the disadvantage of using \acrshort{esb}. 
Using the actors we tried to implement the microservice architecture. In that case, an actor or an actor with slave actors were map to each microservice in \acrshort{wdias}.

/hl{The typical \acrshort{soa} model, for example, usually has more dependent \acrshort{esb}s, with microservices using faster messaging mechanisms. \acrshort{soa} also focuses on imperative programming, whereas microservices architecture focuses on a responsive-actor programming style. Moreover, \acrshort{soa} models tend to have an outsized relational database, while microservices frequently use NoSQL or micro-SQL databases (which can be connected to conventional databases). But the real difference has to do with the architecture methods used to arrive at an integrated set of services in the first place.}
\dbc{Multiple design rounds and decisions are mixed in the para. First finish discussing ESB. Then finish actor model in another para. Then focus on mircoservice details.}

%As mentioned above, \acrshort{esb} support to implement \acrshort{soa}. But \acrshort{soa} has limitations such as single point of failure, slower communication (can’t use for transfer data) etc. So, the microservice architecture get evolved and demanding for the moment. 
When compare to \acrshort{soa}, the microservice architecture has several advantages such as the following:
\begin{itemize}
    \item Follow the Single Responsibility principle
    \item Resilient/flexible as failure in one service does not impact other services. If you have monolithic or bulky service errors in one service/module it can impact other modules/functionality.
    \item High scalable as demanding services can be deployed in multiple servers to enhance performance and keep away from other services so that they do not impact other services. Will be difficult to achieve the same with single, large monolithic service.
    \item Easy to enhance as dependencies are less. Is also easier to change and test.
    \item Low impact on other services as each service is independent. % service., this has less chance to impact other services
    \item Easy to understand since they represent a small piece of functionality
    \item Ease of deployment
\end{itemize}

Even with using the AKKA framework, it has some of the disadvantage as described in AKKA documentation \cite{Akka.ioWhenCluster}. Even Actor model capable of implementing a microservice architecture, it has few disadvantage. To overcome these issues, we moved to the \acrfull{k8s} \cite{LinuxFoundationProduction-GradeKubernetes}.
\dbc{Kubernetes is a product. What you need to refer is to the concept of container-orchestration. Fix.}

\begin{itemize}
    \item \hl{Unable to choose technology. But} \acrshort{k8s} \hl{allows you to choose technology that is best suited for a particular functionality}
    \item \hl{As mentioned in the AKKA documentation} \cite{Akka.ioWhenCluster}, \hl{it is better for developing one microservice within the whole} \acrshort{wdias} design.
\end{itemize}

\dbc{What are these 2 bullets?}
\dbc{Don't use shortened words like it's, isn't, don't,... in writing}