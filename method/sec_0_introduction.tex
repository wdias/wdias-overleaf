This chapter includes the design decision taken while implementing the \acrfull{wdias}, and how is the system got adapted from the systems analyzed in Chapter \ref{ch:literature}, then how is it trying to propose a better system with overcoming the issues.
\dbc{Add a couple of sentences to indicate what is covered in each Section. Do this in all chapters except in Chap 1.}

%%%%%%%%%%%%%%%%%%%%%%%%%%%%%%%%%%%%%%%%%%%%%%%%%%%%%%%%%%%%%%%%%%%%%%%%%%%%%%%%
\section{Introduction}

\dbc{What's this? Are you trying to give a high-kevel idea of the solution. If so, name it as high-level design.}

- Model data
multiple data formats in weather data
GRIB WRF model, out netCDF
netCDF to CSV then feed to HEC-HMS, out CSV
CSV + Rain cell Grid to FLO2D, out water level Grid ASCII files
ASCII files to ArcGIS create Raster files, then use t LOSS Estimation

- Observed data
Automated weather station sending data in higher frequent intervals via HTTP protocol
Automated water level stations send data in higher frequency

Other than above there are many data formats using for handling data based on devices and standards which are using. Also for data modeling, it needs to convert the data into model compatible data format, then after successfully run the model it needs to collect data from the model output data formats.

- Timeseries
A timeseries is simply a series of data points ordered in time. In weather domain, it interest in timeseries in perspective of observations to forecasting.
Each timeseries, the data points can be form in different formats as well.
- Scalar - 0D
- Vector - 1D
- Grid - 2D
- Polygon - 2D

WDIAS is focus on handling Scalar, Vector and Grid data timeseries via the thesis. But the system is capable of extending to handle polygon timeseries as well.

- Base of WDIAS architecture
The base of WDIAS architecture origins based on attributes of a timeseries. There are many attributes to differentiate timeseries one from another. But following can be consider as key attributes to uniquely identify a timeseries from another.
- Module - module that generate the timeseries
- Value Type - Scalar, Vector, Grid, Polygon
- Location 
- Parameter - measuring parameter of timeseries
- Timeseries type
  - sources - External, Simulated
  - category - Historical, forecast
- Time step

- Key components of the system
- Integration
System should be able to store data from multiple sources which come in different formats
- Assimilation
System should be able to provide functionalities for pre-possessing data and validating before using. 
- Dissemination
System should be able to export data in different data formats