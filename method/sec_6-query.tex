\section{Query Timeseries}

As described in \db{Section} \ref{subse:mongodb}, adapter-query is primarily using for support Geo-spatial queries. Other than that, as the primary service for searching for timeseries, it also supports for retrieval of timeseries metadata for external queries as well.

\subsection{Timeseries Metadata}
Get timeseries metadata by timeseriesID.
\begin{lstlisting}
    GET <HOST_NAME>/metadata/timeseries/<TIMESERIES_ID>
\end{lstlisting}
\dbc{Use monospace formatting.}

Search for timeseries metadata with key attributes of the timeseries.
\begin{lstlisting}
    GET <HOST_NAME>/metadata/timeseries?<QUERY_STRING>
\end{lstlisting}
QUERY\_STRING can be consist of zero to multiple parameters. Multiple query parameters are separated by the ampersand, "\&".
\begin{itemize}
    \item moduleId(Optional), e.g., HEC-HMS
    \item valueType(Optional), e.g., Scalar
    \item parameterId(Optional), e.g., O.Precipitation
    \item locationId(Optional), e.g., wdias\_hanwella
    \item timeseriesType(Optional), e.g., ExternalHistorical
    \item timeStepId(Optional), e.g., each\_15\_min
\end{itemize}
For the moment, query are only support by strict matching with above parameter values.

\subsection{Timeseries Geo Queries}

\subsubsection{Query Locations within Area}
Using following API endpoint, it is possible to query the locations which are exists within the area provided by the geoJson data in the JSON payload.
\begin{lstlisting}
    POST <HOST_NAME>/query/location
    JSON Body:
    {
        "geoJson": {
            "type": "Polygon",
            "coordinates": [
                [
                    [
                        <longitude>,
                        <latitude>
                    ],
                    ...
                ]
            ]
        }
    }
\end{lstlisting}

\subsubsection{Query parameters in Locations}
Using the following API endpoint, it is possible to query the parameters stored against given locations. Location IDs should be provide as list of JSON payload.
\begin{lstlisting}
    POST <HOST_NAME>/query/parameter
    JSON Body:
    ["<locationID>", ...]
\end{lstlisting}

\subsubsection{Query Timeseries by Location}
Using the following API endpoint, it is possible to query the timeseries metadata for a given location.
\begin{lstlisting}
    POST <HOST_NAME>/query/timeseries
    JSON Body:
    {
        "location": "<LOCATION_ID>"
    }
\end{lstlisting}

\subsubsection{Query Timeseries by Locations}
Using the following API endpoint, it is possible to query the timeseries metadata for a given location set. Location IDs should be provide as list in the locations field.
\begin{lstlisting}
    POST <HOST_NAME>/query/timeseries
    JSON Body:
    {
        "locations": ["LOCATION_ID", ...]
    }
\end{lstlisting}

\subsubsection{Query Timeseries by Parameter}
Using the following API endpoint, it is possible to query the timeseries metadata by parameter for given location/locations.
\begin{lstlisting}
    POST <HOST_NAME>/query/timeseries
    JSON Body:
    {
        "locations": [""],
        "parameter": ""
    }
\end{lstlisting}

\subsubsection{Query Timeseries in Area}
Using the following API endpoint, it is possible to query the timeseries metadata within a given area. This operation uses geoWithin \acrshort{mongodb} operator for the simplicity.
\begin{lstlisting}
    POST <HOST_NAME>/query/timeseries
    JSON Body:
    {
        "geoJson": {
            "type": "Polygon",
            "coordinates": [
                [
                    [
                        <longitude>,
                        <latitude>
                    ],
                    ...
                ]
            ]
        }
    }
\end{lstlisting}

\subsubsection{Query Timeseries in Area by Parameter}
Using the following API endpoint, it is possible to query the timeseries metadata within a given area for given parameter. This is an extension to the above endpoint.
\begin{lstlisting}
    POST <HOST_NAME>/query/timeseries
    JSON Body:
    {
        "geoJson": {
            "type": "Polygon",
            "coordinates": [
                [
                    [
                        <longitude>,
                        <latitude>
                    ],
                    ...
                ]
            ]
        },
        "parameter": "<PARAMETER_ID>"
    }
\end{lstlisting}

\subsubsection{Query All Timeseries}
Send a request with empty JSON body will respond with all the timeseries.
\begin{lstlisting}
    POST <HOST_NAME>/query/timeseries
    JSON Body:
    {}
\end{lstlisting}
