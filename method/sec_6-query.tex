\section{Query Timeseries}
\label{se:query}

As described in \cref{subse:mongodb}, adapter-query is primarily using to support Geospatial queries. Other than that, as the primary service for searching for timeseries, it also supports for retrieval of timeseries metadata for external queries as well.

\subsection{Timeseries Metadata}
Get timeseries metadata by timeseriesID.
\begin{lstlisting}
    GET <HOST_ADDR>/metadata/timeseries/<TIMESERIES_ID>
\end{lstlisting}

Search for timeseries metadata with key attributes of the timeseries.
\begin{lstlisting}
    GET <HOST_ADDR>/metadata/timeseries?<QUERY_STRING}
\end{lstlisting}

QUERY\_STRING can be consist of zero to multiple parameters. Multiple query parameters are separated by the ampersand, "\&".
\begin{itemize}
    \item \texttt{moduleId} (Optional), e.g., HEC-HMS
    \item \texttt{valueType} (Optional), e.g., Scalar
    \item \texttt{parameterId} (Optional), e.g., O.Precipitation
    \item \texttt{locationId} (Optional), e.g., wdias\_hanwella
    \item \texttt{timeseriesType} (Optional), e.g., ExternalHistorical
    \item \texttt{timeStepId} (Optional), e.g., each\_15\_min
\end{itemize}
For the moment, query are only support by strict matching with above parameter values.

\subsection{Timeseries Geo Queries}

\paragraph{Query Locations within Area}-- Using the following API endpoint, it is possible to query the locations which exist within the area provided by the GeoJson data in the JSON payload.
\begin{lstlisting}
    POST <HOST_ADDR>/query/location
    JSON Body:
    {
        "geoJson": {
            "type": "Polygon",
            "coordinates": [
                [
                    [
                        <longitude>,
                        <latitude>
                    ],
                    ...
                ]
            ]
        }
    }
\end{lstlisting}

paragraph{Query parameters in Locations}-- Using the following API endpoint, it is possible to query the parameters stored against given locations. Location IDs should be provided as a list of JSON payload.
\begin{lstlisting}
    POST <HOST_ADDR>/query/parameter
    JSON Body:
    ["<locationID>", ...]
\end{lstlisting}

\paragraph{Query Timeseries by Location}-- Using the following API endpoint, it is possible to query the timeseries metadata for a given location.
\begin{lstlisting}
    POST <HOST_ADDR>/query/timeseries
    JSON Body:
    {
        "location": "<LOCATION_ID>"
    }
\end{lstlisting}

\subsubsection{Query Timeseries by Locations}
Using the following API endpoint, it is possible to query the timeseries metadata for a given location set. Location IDs should be provide as list in the locations field.
\begin{lstlisting}
    POST <HOST_ADDR>/query/timeseries
    JSON Body:
    {
        "locations": ["LOCATION_ID", ...]
    }
\end{lstlisting}

\subsubsection{Query Timeseries by Parameter}
Using the following API endpoint, it is possible to query the timeseries metadata by parameter for given location/locations.
\begin{lstlisting}
    POST <HOST_ADDR>/query/timeseries
    JSON Body:
    {
        "locations": [""],
        "parameter": ""
    }
\end{lstlisting}

\paragraph{Query Timeseries in Area} -- Using the following API endpoint, it is possible to query the timeseries metadata within a given area. This operation uses geoWithin  MongoDB operator for the simplicity.
\begin{lstlisting}
    POST <HOST_ADDR>/query/timeseries
    JSON Body:
    {
        "geoJson": {
            "type": "Polygon",
            "coordinates": [
                [
                    [
                        <longitude>,
                        <latitude>
                    ],
                    ...
                ]
            ]
        }
    }
\end{lstlisting}

\paragraph{Query Timeseries in Area by Parameter}-- Using the following API endpoint, it is possible to query the timeseries metadata within a given area for given parameter. This is an extension to the above endpoint.
\begin{lstlisting}
    POST <HOST_ADDR>/query/timeseries
    JSON Body:
    {
        "geoJson": {
            "type": "Polygon",
            "coordinates": [
                [
                    [
                        <longitude>,
                        <latitude>
                    ],
                    ...
                ]
            ]
        },
        "parameter": "<PARAMETER_ID>"
    }
\end{lstlisting}

\paragraph{Query All Timeseries} -- Send a request with empty JSON body will respond with all the timeseries.
\begin{lstlisting}
    POST <HOST_ADDR>/query/timeseries
    JSON Body:
    {}
\end{lstlisting}

\dbc{Need a chapter summary Sec.}