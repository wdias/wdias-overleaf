\section{Query Timeseries}
\label{se:query}

We used the adapter-query primarily for supporting Geospatial queries. As the primary service for searching for timeseries, it supports retrieval of timeseries metadata for external queries as well. The following sections described the search query capabilities of \acrshort{wdias}.

%%%%%%%%%%%%%%%%%%%%%%%%%%%%%%%%%%%%%
\subsection{Timeseries Metadata}
Get timeseries metadata by timeseriesID.
\begin{lstlisting}
    GET <HOST_ADDR>/metadata/timeseries/<TIMESERIES_ID>
\end{lstlisting}

Search for timeseries metadata with key attributes of the timeseries.
\begin{lstlisting}
    GET <HOST_ADDR>/metadata/timeseries?<QUERY_STRING}
\end{lstlisting}

QUERY\_STRING can be consist of zero to multiple parameters. Multiple query parameters are separated by the ampersand, "\&".
\begin{itemize}
    \item \texttt{moduleId} (Optional), e.g., HEC-HMS
    \item \texttt{valueType} (Optional), e.g., Scalar
    \item \texttt{parameterId} (Optional), e.g., O.Precipitation
    \item \texttt{locationId} (Optional), e.g., wdias\_hanwella
    \item \texttt{timeseriesType} (Optional), e.g., ExternalHistorical
    \item \texttt{timeStepId} (Optional), e.g., each\_15\_min
\end{itemize}
For the moment, query are only support by strict matching with above parameter values.

%%%%%%%%%%%%%%%%%%%%%%%%%%%%%%%%%%%%%
\subsection{Timeseries Geo Queries}

\paragraph{Query Locations within Area}-- Following endpoint supports to query the locations which exist within the area provided by the GeoJson data in the JSON payload.
\begin{lstlisting}
    POST <HOST_ADDR>/query/location
    JSON Body:
    {
        "geoJson": {
            "type": "Polygon",
            "coordinates": [
                [
                    [
                        <longitude>,
                        <latitude>
                    ],
                    ...
                ]
            ]
        }
    }
\end{lstlisting}

paragraph{Query parameters in Locations}-- Following endpoint supports to query the parameters stored against given locations. Location IDs should be provided as a list of JSON payload.payload.
\begin{lstlisting}
    POST <HOST_ADDR>/query/parameter
    JSON Body:
    ["<locationID>", ...]
\end{lstlisting}

\paragraph{Query Timeseries by Location}-- Following endpoint supports to query the timeseries metadata for a given location.
\begin{lstlisting}
    POST <HOST_ADDR>/query/timeseries
    JSON Body:
    {
        "location": "<LOCATION_ID>"
    }
\end{lstlisting}

\subsubsection{Query Timeseries by Locations}
Following endpoint supports to query the timeseries metadata for a given location set. Location IDs should be provided as a list in the locations field.
\begin{lstlisting}
    POST <HOST_ADDR>/query/timeseries
    JSON Body:
    {
        "locations": ["LOCATION_ID", ...]
    }
\end{lstlisting}

\subsubsection{Query Timeseries by Parameter}
Following endpoint supports to query the timeseries metadata by parameter for given location/locations.
\begin{lstlisting}
    POST <HOST_ADDR>/query/timeseries
    JSON Body:
    {
        "locations": [""],
        "parameter": ""
    }
\end{lstlisting}

\paragraph{Query Timeseries in Area} -- Following endpoint supports to query the timeseries metadata within a given area (specified as a polygon in geoJson format).
\begin{lstlisting}
    POST <HOST_ADDR>/query/timeseries
    JSON Body:
    {
        "geoJson": {
            "type": "Polygon",
            "coordinates": [
                [
                    [
                        <longitude>,
                        <latitude>
                    ],
                    ...
                ]
            ]
        }
    }
\end{lstlisting}

\paragraph{Query Timeseries in Area by Parameter}-- Following endpoint supports to query the timeseries metadata within a given area for a given parameter. This endpoint is an extension of the above endpoint.
\begin{lstlisting}
    POST <HOST_ADDR>/query/timeseries
    JSON Body:
    {
        "geoJson": {
            "type": "Polygon",
            "coordinates": [
                [
                    [
                        <longitude>,
                        <latitude>
                    ],
                    ...
                ]
            ]
        },
        "parameter": "<PARAMETER_ID>"
    }
\end{lstlisting}

\paragraph{Query All Timeseries} -- Send a request with an empty JSON body that will respond with all the timeseries.
\begin{lstlisting}
    POST <HOST_ADDR>/query/timeseries
    JSON Body:
    {}
\end{lstlisting}
