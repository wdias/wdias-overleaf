\section{Summary}
\label{se:method_summary}

Throughout this Chapter, it discussed about the methodology of designing and implementing the \acrshort{wdias} based on the concepts studied on \cref{ch:literature}. It uses microservice architecture over using \acrshort{soa} with a \acrshort{esb} to avoid the draw back of streaming bulk data over an \acrshort{esb}. In order to take the full advantage of using microservice architecture, we moved to container orchestration based architecture rather building tightly coupled microservices with actor model.
While implementing the import and export modules, those modules follows the architecture style of \emph{Smart Endpoints and Dumb Pipes} and many other places. It implement each microservice as a containerized application which can deploy independent of other microservices. Also it uses the pattern of \emph{database per service} with combining different type of database systems to create the \acrshort{wdias} database structure. Further it provides a comprehensive extension modules system which allow users to preprocess data with creating new trigger without any downtime for system configuration. It indexing mechanism with Geo query support most of the required queries to search for any timeseries data.
