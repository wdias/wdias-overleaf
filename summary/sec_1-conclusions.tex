%\section{Summary}
\section{Conclusions}
\dbc{Expand this section while highlighting key design details and performance}
\dbc{Sec. 5.2 should be "Research Limitations"}

To enhance the accuracy of the weather prediction, it is required to have reliable weather data for \acrshort{nwm}. \db{Most of existing weather data systems are proprietary or close sourced, or not up to date with the existing cloud computing technologies.}
We proposed an extendable open source Weather Data Integration and Assimilation Systems which is known as \acrshort{wdias}. It focused on providing efficiently integrates weather data from different sources with quality control and supporting steaming large size of data. And compatible with current cloud computing technologies and architecture patterns.
\dbc{fix highlighted above}

First the architecture starts with SOA using a \acrfull{esb}, and then moved to Actor model based architecture. After comparing the disadvantages of using such architecture with the system requirements, final architecture come up with modern microservice architecture.
Based on the microservice architecture patterns and the nature of the weather metadata, the \acrshort{wdias} came up with the idea of hierarchical database structure in order to provide higher performance while store the weather data. The system uses \acrshort{influxdb} timeseries database for storing Scalar and Vector timeseries data, and using \acrshort{netCDF} for storing Grid timeseries data.
Also the system provide a generic open mechanism to integrate new modules as extension in order to enhance the features of the system. This capability enable to integrate weather data preprocessing flows as extension modules.
The extension API provide easy access to create and modify the extension triggers on the fly without stopping the system or any downtime in order to change the configurations.
The system provides extensive timeseries query endpoints to easy search over the system timeseries metadata with supporting Geo based queries.

\acrshort{wdias} performance test is perform using the \acrshort{jmeter} tool's distributed testing capabilities which separately testing each modules. Then the test plans are performed with increasing the request size, and monitored the performance metrics. The system is setup on the \acrfull{eks} as per the configurations of \ref{sub:test_sys_config}.
During the performance test plans, the system were able to provide constant throughput by keeping the latency constant while increasing the work load. Which exhibit the scalability of the system.
With the \acrshort{k8s} auto scaling, the \acrshort{wdias} were able to elastically adjust the number of pods according to the work load with the given configurations.
