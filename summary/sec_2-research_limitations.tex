\section{Research Limitations}
\label{se:research_limitations}

Next, we discussed \emph{Lack of data preprocessing modules}:
We implemented the extension modules as an eco-system of integrating different preprocessing modules as add-ons to the \acrshort{wdias}. Then, users can use these modules for process the data which are inserted into the system in real time or process later based on regular intervals. Also, we provide a more generic open interface approach to create more preprocessing modules and trigger them via the extension API. Since \acrshort{wdias} is an open-source system, we are expecting that the community will create more modules, and provide them as contributions for future users. When compared to other systems like \acrshort{fews}, the current system only has few extensions for the testing purpose. Thus, we can consider that as one of the major areas that need to be improved.

\emph{Grid data performance}:
As mentioned in \cref{se:discussion}, grid adapter is using a Python language wrapper for netCDF FORTRAN implementation with parallel IO enabled. However, it has some performance issues and memory leak issues as well. Since each microservice in the \acrshort{wdias} is independent of technology, we can use a different technology stack to implement the service. As an example, we can use a low-level language such as C or FORTRAN to implement the grid adapter, which will affect performance improvements and lesser issues. There is some future work that can focus on improving the performance of storing grid data.

\emph{Improvements to test cases}:
As described in detail with \cref{subse:closed_vs_open_workload}, the \acrshort{wdias} test plans use the time-stepping feedback loop with concurrent threads, which is the best approach support by JMeter at the moment. However, using the above approach, we cannot get the desired \acrshort{rps} for each test case within the test plan.
Because of the above issue, the \acrshort{wdias} test plans further improved by removing redundant test cases such as create timeseries at the beginning. Having created timeseries at the beginning causes some issues such as most of the thread are working on that test case, rather than running the required test cases. Thus, the test plans need to be further improved with running more test cases with the desired \acrshort{rps}.

\emph{Tune \acrshort{wdias} database structure performance}:
As further described in \cref{se:discussion}, we can increase the performance of adapter microservices by using InfluxDB commercial cluster support for high availability or horizontal scaling of InfluxDB. With the InfluxDB cluster, it is possible to run multiple pods of scalar adapter and vector adapter to support more server hits per second. The \acrshort{wdias} performance can improve via partitioning the timeseries key space into multiple InfluxDB instances. As an example, based on the key attribute \cref{subse:timeseries_key_attributes} Timeseries Type, the scalar adapter can connect to four instances of InfluxDB instances such as external historical, external forecast, simulated historical, and simulated forecast.

\emph{Explicitly not testing for high availability}:
\gkc{By default, the container orchestration framework is handling service failures, auto scaling, auto down scaling, and independent service deployments. Also, when considering about modern cloud resources in data centers have better availability as well and failure handling. Combining both of factors will enable high availability for the \acrshort{wdias} system. But in this research, we are not heavily rely on those third party tools. Also, we do not explicitly test for the high availability with performing manually terminating services, and letting crash some of the services to measure how is it affect on the \acrshort{wdias} system high availability.}

%%%%%%%%%%%%%%%%%%%%%%%%%%%%%%%%%%%%%%%%%%%%%%%%%%%%%%%%%%%%%%%%%%%%%%%%%%%%%%%%%%
\section{Future Work}
\label{se:future_work}
Here, we discussed the enhancements that we can add to improve the performance and usability of the system.

\emph{Supporting irregular grids}:
\gkc{When consider about 2D grid, there are two types of grids such as regular grids and irregular grids. In regular grid system, each shell consist of fix width and height. But for irregular grid system, each shell define as a 2D polygon with there's coordinates.}
At the moment, \acrshort{wdias} does not support irregular grids, but it has the API endpoints defined for the implementation of such support with following the microservice architecture. We reduced the scope of the research by removing such components from the system, and such level of implementation does not belong to the \acrshort{wdias} scope as well.

\emph{Define infrastructure as code}:
Physical nodes for the \acrshort{k8s} cluster can be created using cloud computing providers such as Amazon EKS, Google Cloud, Microsoft Azure cloud, or even using own cluster. \gkc{During the load testing, we used Amazon cloud based physical machines as nodes for the test setup. And users need to have some basic understanding on Amazon cloud and it's technologies to setup the system, and the \acrshort{k8s} cluster}. However, the infrastructure that needs to deploy the \acrshort{wdias} can be defined as code such as using tools like Terraform, which is independent of the cloud computing provider. Using such a tool, users will be able to quickly deploy the \acrshort{wdias} on any cloud provider without much hassle. \gkc{The \acrshort{wdias} resource requirements are configured as code in the Terraform, and user can easily interact with those clouds using Terraform provider modules without much prior knowledge.}

\emph{Alerting extension modules}: At the moment in the \acrshort{wdias} basic framework, the extensions only support OnChange and OnTime event triggers. However, it is useful to have event triggers support based on the configured alert level or threshold level. \gkc{Here, users should be able to register triggers based on given alert levels, and the alerting extension should trigger a webhook which is provide by the user while registering the trigger. As an example, a user can register an alert event on particular timeseries such as waterlevel, and also provide the threshold value based on historical flood events. When the waterlevel exceeds the given threshold value, the system will trigger the provided webhook. Then the user will be notified or the webhook endpoint will take necessary actions.}

\emph{Publisher Subscriber extension modules}: The \acrshort{wdias} better to be capable of triggering external systems based via Publisher Subscriber support. \gkc{As an example, we may required to sync weather station's precipitation timeseries data into an external system through \acrshort{wdias} when there is a rain. One possibility is, the external system can sync data with timeseries data via polling through the \acrshort{wdias}. But it is inefficient, since the external system needs to fetch the data via export modules even there is not any rain. Rather, we can use Publisher Subscriber capabilities to efficiently sync the data. With this extension modules, users can create topics and publishers extensions will push data into the topics when particular criteria is met, such as when there is rain in above example. Then external systems will subscribes to topic above topics while providing a webhook endpoint. Whenever there is data pushed into the topic, the extension modules will send data to the subscribers via webhooks.}
