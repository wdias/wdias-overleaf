\section{Research Limitations}
\label{se:research_limitations}
\gkc{Previously you asked to rename this section sir. Why can't it have the title of \emph{Future Work}. There are some of the requirements that can be done for the system, in order to gain more of it. Which part is misleading?}

This section discussed the lacking features in the \acrshort{wdias} and some of the improvements that can add to improve the performance and usability of the system.

\emph{Lack of data preprocessing modules}:
The extension module enables the plugin system to integrate prepossessing modules that can use for process the data which are inserted to the system or based on regular intervals. Also, \acrshort{wdias} provides a more generic open interface approach to create more preprocessing modules. Since \acrshort{wdias} is an open-source system, it expects more modules to be created by the community. When compared to other systems like \acrshort{fews}, the current system only have few extension for the testing purpose. This is one of the major areas that need to be improved.

\emph{Supporting Irregular Grids}:
At the moment \acrshort{wdias} does not support Irregular Grids, but it has the API endpoints defined for the implementation which can follow the same microservice architecture. The scope of the \acrshort{wdias} does not belong to such level of implementation and the scope was reduced by removing such components from the system.

\emph{Grid data performance}:
As mentioned in the \cref{se:discussion}, adapter-grid is using a Python wrapper for netCDF FORTRAN implementation with parallel IO enabled. This has some performance issues as well as memory leak issues. Since each microservice in the \acrshort{wdias} independent of technology that can use to implement the service, it will increase the performance if the adapter-grid directly ported to netCDF with a low-level language such as C or FORTRAN. They are some future work that can focus on improving the performance of grid data.

\emph{Define infrastructure as code}:
Physical nodes for the \acrshort{k8s} cluster created using Cloud Computing provider such as \acrshort{eks}. And defining each pod resource limitations at the helm chart level. But the infrastructure that needs to deploy the \acrshort{wdias} can define as code such as using tools like Terraform which is independent of the Cloud Computing provider. Using such a tool, users will able to easily deploy the \acrshort{wdias} on any Cloud provider without much hassle.

\emph{Improve test cases}:
As described in detail with \cref{subse:closed_vs_open_workload}, for \acrshort{wdias} performance test uses the \emph{Time Stepping Feedback} loop with Concurrent Threads which is the best approach support by  JMeter at the moment. But using that it is not possible to get the desired \acrshort{rps} for each test case within the test plan.
Because of this issue, the \acrshort{wdias} test plans further improved by removing redundant test cases such as create timeseries at the beginning. Having created timeseries at the beginning causes some issues such as most of the thread are working on that test case rather on required test cases. Still, there may be some improvement that can be done. 

\emph{Tune \acrshort{wdias} database structure performance}:
As further described in \cref{se:discussion}, it is possible to increase the performance of these bottlenecks much higher such as using  InfluxDB Commercial cluster support for high availability or horizontal scaling of  InfluxDB. With the  InfluxDB cluster, it is possible to run multiple pods of adapter-scalar and adapter-vector for support more server hits per second.
Other than that, the \acrshort{wdias} performance can improve via partitioning the timeseries key space into multiple  InfluxDB instances. The adapter-scalar or adapter-vector can improve to plug multiple instances of  InfluxDB instances by changing the configurations. As an example, based on the key attribute \cref{subse:timeseries_key_attributes} Timeseries Type, the adapter-scalar can connect to four instances of  InfluxDB instances such as  External Historical, External Forecast, Simulated Historical, and Simulated Forecast, etc.

\emph{Alerting extension modules}:
At the moment, extensions only support \emph{OnChange} and \emph{OnTime} events. But it will be good to have event trigger support based on reconfigured alert level.

\emph{Publisher Subscriber extension modules}:
The \acrshort{wdias} better to be capable of triggering external systems based on events such as webhooks.

\dbc{Need a new section for Conclusions.}
\gkc{It did not make sense for me to adding another section as Conclusion. We already add a \cref{se:summary_conclusion} above. Do we need to rearrange the content of this Chapter? No point of adding another section with repeating the same content. May be I didn't understand your suggestion here.}
\gkc{Do we need to move the \cref{se:discussion} here? Normally where should we include the analysis of observations?}