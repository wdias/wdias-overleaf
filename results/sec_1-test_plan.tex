\section{Test Plan}
The system consists of three modules in functionality perspective of interacting with the \acrshort{wdias}.
\begin{enumerate}
    \item Import modules
    \item Export modules
    \item Extension modules
\end{enumerate}
Other than above, adapter modules act as the core modules and the base of the system. Also query modules allow the users to query for the timeseries
based on the metadata, and enable performing Geo based queries based on the locations.

The test plan is to perform the load test on each above module as separate entity in order to analyse the individual module scalability.
Then perform load testing on whole system, and analyse the scalability in the system. There are two variables that can vary while doing the performance testing.
\begin{enumerate}
    \item Number of requests that can handle at a given time (\acrfull{rps})
    \item Request size
\end{enumerate}
A timeseries is a series of data points in time order. The system may receive the data points in multiple requests.
The request size can vary based on the use case from one data point to multiple data points. As few examples;
\begin{itemize}
    \item A weather station may send current water level in a regular period of 1 minute. In such case, it always comes as a request with one data point with higher frequency.
    \item Due to resource issue such as batteries, a weather station can collect data points for a given period of time and send as a batch. Another example would be,
A weather station can collect the precipitation at each 5 minute period and send all the data in 1 hour intervals.
    \item The forecast precipitation data from \acrshort{wrf} model can be extract for a day in one hour interval and store in the system.
\end{itemize}

The percentage of each data type can be vary depend on the situation. But most of the applications collect many data from the weather stations in order to correct and validate the models data. These weather stations can send the data in different frequencies with different parameters such as precipitation, temperature, humidity, wind direction etc.
Most of these are scalar data. Parameters such as wind direct are vector data which has less percentage compared to scalar data. On other side, most of the grid data generate from models and simulations. Those produce in less frequency due to cost of the computer resource to produce results. After considering those factor, it is logical to come with following percentages;
\begin{itemize}
    \item Scalar - 70\%
    \item Vector - 20\%
    \item Grid - 10\%
\end{itemize}

As a upper limit of such scenarios while designing the performance testing, it uses following;
\begin{itemize}
    \item Hourly (24 data points) 
    \item Every 30 minutes (48 data points)
    \item Every 5 minutes (288 data points) - One request is equivalent to a data points send by a weather station in 5 minutes interval for a whole day
\end{itemize}

\subsection{Test Plan Flow}
\label{subse:test_plan_flow}
Following are the test cases which are planned to performance test cases;
\begin{enumerate}
    \item Test Setup to create timeseries in 1hr, 30min and 15min intervals and create metadata of the timeseries
    \item Import Timeseries
    \begin{itemize}
        \item Plan for test run of 15 minutes with the request size of 15min data.(total 0.25 hours)
        \item Have mixture of data: Scalar - 70\%(max 210 requests per second), Vector (Multi-Scalar) - 20\% (max 60 requests per second), Grid - 10\% (max 30 requests per second)
    \end{itemize}
    \item Extensions
    \begin{itemize}
        \item Create Extensions for /Aggregation\_Accumulative, /Interpolation Linear, /Validation Missing Values (OnChange and OnTime)
        \item Plan for the 15 minutes of test run with the request size of 15min data. (total 0.25 hours)
        \item Do with Import Timeseries with error data which will go through extensions.
    \end{itemize}
    \item Export Timeseries
    \begin{itemize}
        \item Plan for 15 minutes of test run with the request size of 15min data. (total 0.25 hours)
        \item Have mixture of data: Scalar - 70\%, Vector (Multi-Scalar) - 20\%, Grid - 10\% (Use the Imported - Data before and verify against Extensions)
    \end{itemize}
    \item Import + Extension + Export + Timeseries Queries (All)
    \begin{itemize}
        \item Plan for 30 minutes of test run with the request size of 1hr (60min) data.
        \item Plan for 30 minutes of test run with the request size of 30min data.
        \item Plan for 30 minutes of test run with the request size of 15min data.
        \item Plan for 30 minutes of test run with the request size of 15min data with Auto Scaling.
        \item All test plan run against 60min, 30min and 15min request size of data. Another test plan run with enabling the Auto Scaling. (total 2 hours)
    \end{itemize}
\end{enumerate}

\subsection{Test Plan Performance Metrics}
\label{subse:test_plan_metrics}
During the performance analysis, following metrics are use to measure the performance of the system.
\begin{itemize}
    \item Throughput - Number of requests that can be process per unit time. Normally for \acrshort{wdias} performance observations, it uses \acrfull{rps} in order to measure the throughput.
    \item Latency - Time taken for response back to the sent request to the system. Normally it measures with milliseconds (ms) in the \acrshort{wdias}. During the performance analysis, it measures the minimum response time, maximum response time, the average response time, and standard deviation. The standard deviation measures the mean distance of the values to their average.It gives an idea of the dispersion or variability of the measures to their mean value. So, if the deviation value is low compared to the mean value, it will indicate that the measures are not dispersed (or mostly close to the mean value) and that the mean value is significant. Other than that, it measure the response time into 90\% percentile, while gives best response time up to 90\% percent of the responses.
    \item Resource Utilization - Resource Utilization is calculated based on the CPU and Memory usage via \acrshort{k8s} metrics server. It calculates the resource utilization over 60 seconds of time window. Network usage is getting based on the actual physical machines.
    \item Auto Scaling - The \acrshort{wdias} is running with two configurations. First is, since it is running the microservice as docker containers, it allow to increase or shrink the resource usage and share the resources within a physical machine. Second is, running with auto scaling enabled for the microservice which are using higher resources.
\end{itemize}
