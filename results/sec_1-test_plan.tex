\section{Test Plan}
The system consists of three modules in functionality perspective of interacting with the \acrshort{wdias}.
\begin{enumerate}
    \item Import modules
    \item Export modules
    \item Extension modules
\end{enumerate}
Other than above, adapter modules act as the core modules and the base of the system. Also query modules allow the users to query for the timeseries
based on the metadata, and enable performing Geo based queries based on the locations.

The test plan is to perform the load test on each above module as separate entity in order to analyse the individual module scalability.
Then perform load testing on whole system, and analyse the scalability in the system. There are two variables that can vary while doing the performance testing.
\begin{enumerate}
    \item Number of requests that can handle at a given time (\acrfull{rps})
    \item Request size
\end{enumerate}
A timeseries is a series of data points in time order. The system may receive the data points in multiple requests.
The request size can vary based on the use case from one data point to multiple data points. As few examples;
\begin{itemize}
    \item A weather station may send current water level in a regular period of 1 minute. In such case, it always comes as a request with one data point with higher frequency.
    \item Due to resource issue such as batteries, a weather station can collect data points for a given period of time and send as a batch. Another example would be,
A weather station can collect the precipitation at each 5 minute period and send all the data in 1 hour intervals.
    \item The forecast precipitation data from \acrshort{wrf} model can be extract for a day in one hour interval and store in the system.
\end{itemize}

As a upper limit of such scenarios while designing the performance testing, it uses following;
\begin{itemize}
    \item Hourly (24 data points) 
    \item Every 30 minutes (48 data points)
    \item Every 5 minutes (288 data points) - One request is equivalent to a data points send by a weather station in 5 minutes interval for a whole day
\end{itemize}

Following are the test cases which are planned to perform;
\begin{enumerate}
    \item Test Setup to create timeseries in 1hr, 30min and 5min intervals and create metadata of the timeseries
    \item Import Timeseries
    \begin{itemize}
        \item Plan for test run of 30 minutes. In each test run change the request size from 1hr, 30min and 5min. (total 1.5 hours)
        \item Have mixture of data: Scalar - 70%(max 210 requests), Vector (Multi-Scalar) - 20% (max 60 requests), Grid - 10% (max 30 requests)
    \end{itemize}
    \item Extensions
    \begin{itemize}
        \item Create Extensions for /Aggregation\_Accumulative, /Interpolation Linear, /Validation Missing Values (OnChange and OnTime)
        \item 30min of test run. Just for 5min data. (total 0.5 hours)
        \item Do with Import Timeseries with error data which will go through extensions.
    \end{itemize}
    \item Export Timeseries
    \begin{itemize}
        \item 30min test run. Change request size from 1hr, 30min and 5min. (total 1.5 hours)
        \item Have mixture of data: Scalar - 70%, Vector (Multi-Scalar) - 20%, Grid - 10% (Use the Imported - Data before and verify against Extensions)
    \end{itemize}
    \item Import + Extension + Export + Timeseries Queries (All)
    \begin{itemize}
        \item 30min test run. Change with request size from 1hr, 30min and 5min. (total 1.5 hours)
    \end{itemize}
\end{enumerate}

Throughput
Latency
Resource Utilization
Auto Scaling