\documentclass[a4paper,oneside,12pt]{report}
\usepackage{graphicx,mathptmx}
\usepackage{geometry}
 \geometry{
 a4paper,
 %total={160mm,227mm},
 left=40mm,
 top=25mm,
 bottom=40mm, 
 right=25mm,
}
\renewcommand{\baselinestretch}{1.5} 

\usepackage{setspace}
\usepackage{tabu}
\usepackage[acronym]{glossaries}
\makeglossaries
\usepackage{import}
\usepackage{listings}
\usepackage{xcolor}
\usepackage{dirtytalk}
\usepackage{tabulary}
\usepackage{biblatex}

\usepackage{todonotes}
\newcommand{\db}[1]{\textcolor{blue!40}{#1}}
\newcommand{\dbc}[1]{\todo[author=Dilum, inline, color=blue!40]{#1}}

\addbibresource{mendeley.bib}

\definecolor{codegreen}{rgb}{0,0.6,0}
\definecolor{backcolour}{rgb}{0.95,0.95,0.92}
\lstdefinestyle{code-style}{
  backgroundcolor=\color{backcolour}, commentstyle=\color{codegreen},
  basicstyle=\ttfamily\footnotesize,
  breakatwhitespace=false,         
  breaklines=true,                 
  captionpos=b,                    
  keepspaces=true,                 
  numbers=left,                    
  numbersep=5pt,
  showspaces=false,
  showstringspaces=false,
  showtabs=false,                  
  tabsize=2
}
\lstset{style=code-style}

\usepackage{newfloat} % Caption Lists: https://tex.stackexchange.com/a/239114/181545
\DeclareFloatingEnvironment[placement={!ht},name=List]{customlist}

\begin{document}
\begin{titlepage}
    \begin{center}
        \Large{
        \textbf{WEATHER DATA INTEGRATION AND ASSIMILATION SYSTEM}}\\
        \vspace{144pt}
  \large      
        %\vspace{1cm}
      % by
       
        
        %\vspace{1.0cm}
        
        Gihan Chanuka Karunarathne\\
        \vspace{24pt}
       % \normalsize
      178004U\\
         \vspace{72pt}
        %\normalsize
        Degree of Master of Science\\
       
        %\centering
         %\includegraphics[width=0.4\textwidth]{0.Title_Page/uom.png}\\
            %\vspace{0.8cm}
         %\normalsize
        
        %\vfill
       \vspace{72pt}
        \large
        Department of Computer Science and Engineering\\
        \vspace{24pt}
        University of Moratuwa\\
        Sri Lanka\\
        \vspace{32pt}
        February 2020
        
    \end{center}
\end{titlepage}

\begin{titlepage}
    \begin{center}
       % \vspace*{1cm}
        \Large{
        \textbf{WEATHER DATA INTEGRATION AND ASSIMILATION SYSTEM}}\\
        \vspace{144pt}
  \large      

        Herath Mudiyanselage Gihan Chanuka Karunarathne\\
        \vspace{24pt}
       % \normalsize
        178008U\\
         \vspace{72pt}
        \normalsize
        Thesis submitted in partial fulfillment of the requirements for the degree Master of Science in Computer Science and Engineering\\
     
        %\vfill
       \vspace{72pt}
        \large
        Department of Computer Science and Engineering\\
        \vspace{24pt}
        University of Moratuwa\\
        Sri Lanka\\
        \vspace{32pt}
        February 2020
        
    \end{center}
\end{titlepage}

\pagenumbering{roman}

\newgeometry{left=4cm,top=0.5cm,right=2.5cm}
\chapter*{Declaration}
\addcontentsline{toc}{chapter}{Declaration}

I declare that this is my own work and this thesis does not
incorporate without acknowledgement any material previously submitted for a
Degree or Diploma in any other University or institute of higher learning and to
the best of my knowledge and belief it does not contain any material previously
published or written by another person except where the acknowledgement is
made in the text.

Also, I hereby grant to University of Moratuwa the non-exclusive right to
reproduce and distribute my thesis/dissertation, in whole or in part in print,
electronic or other medium. I retain the right to use this content in whole or part
in future works (such as articles or books).

\vspace{0.5in}
\noindent
\begin{tabu} to 1.0\textwidth { X[l] X[l] }
    Signature: & Date:
\end{tabu}



\vspace{0.5in}
\noindent
The above candidate has carried out research for the Masters thesis under our supervision.


\vspace{0.5in}
\noindent
\begin{tabu} to 1.0\textwidth { X[l] X[l] }
    Name of the supervisor: & Dr. HMN Dilum Bandara\\ [1.5ex]
    Signature of the supervisor: & Date:
\end{tabu}

\restoregeometry
\normalsize


\newgeometry{left=2.5cm,top=0.2cm,right=2cm}
\addcontentsline{toc}{chapter}{Abstract} 
{\setstretch{1.0} 
\chapter*{Abstract}
%\chapter*{\hspace{1cm} Abstract}

\acrfull{nwm} utilize data collected via diverse sources such as automated weather stations, radars, air balloons, and satellite images. Prior to using such multimodal data into respective \acrshort{nwm}, it is necessary to transcode into a format that can be ingested by the \acrshort{nwm} modal. Moreover, the data integration system's response time needs to be relatively low to forecast and monitor time-sensitive weather events like hurricanes, storms, and flash floods that require rapid and frequent execution of \acrshort{nwm}. Later such weather data needs to be accessed by many third-party applications and research in order to make use of them. Thus the system should provide easy protocol to integrate with external systems, and should be capable of handling the workload. For example, logistic companies could use those data within their models to plan and schedule deliveries. Agricultural insurance companies can warn the farmers in advance, as well as calculate premiums based on anticipated weather patterns.

Even though there are many data integration systems, most of them are proprietary or close sourced. Which means for an island like Sri Lanka that having different kinds of weather seasons over the year than those software originated need to be highly customized. Throughout the operations and maintenance, it is required to get support with paying lots of charges, and difficult to debug into a closed source system without maintenance support from the vendors. And most of the software existing out there is not up to date with the latest technology concepts such as Cloud Computing. \acrfull{wdias} uses the modern architecture pattern such as microservice in order to achieve scalability, rather using monolithic architecture or client service architecture.

It provides a modular approach to integration of data from different sources and export into different formats, and inbuilt extension module system allow users to add new features. Since each module follows the stateless and failover microservice integrate with new Cloud Computing tools, the proposed solution comes with the scalability and high availability into large extend out of the box. The proposed solution is independent of the platform which most software out there is not and independent of Cloud Platform providers which give much flexibility for the users to decide the best cost effective way to host it. And the auto scaling feature allows users to run \acrshort{wdias} from 1 CPU node to nodes with a few hundred CPUs. Unlike most existing systems \acrshort{wdias} does not require a shutdown in order to update the system configuration, and it allows to add new features on the fly with rollover capabilities.

\vspace{4mm}

\textbf{Keywords:} Cloud Computing, Data Management system, Distributed,  Highly Available, Scalable, Timeseries, Weather
}
\restoregeometry
\normalsize

\addcontentsline{toc}{chapter}{Dedication} 
\chapter*{Dedication}
I dedicate my thesis work to my family, teachers and specially friends at \acrfull{curw}. A special feeling of gratitude to my loving parents, Nandawathi Dissanayake and H.M.K. Karunarathne whose words of encouragement and push for tenacity ring in my ears. My wife Jayani Kumarasinghe have never left my side and are very special.

I also dedicate this thesis to my many friends at \acrshort{curw} who have supported and be with me during this period of time. Also special thanks to Dr. Dilum Bandara and Dr. Srikantha Herath for giving me this wonderful opportunity to explore this new domain.

\addcontentsline{toc}{chapter}{Acknowledgements} 
\chapter*{Acknowledgements}
I wish to thank my evaluation panel members who were more than generous with their expertise and precious time. A special thanks to Dr. Dilum Bandara, my research supervisor for his countless hours of reflecting, reading, encouraging, and most of all patience throughout the entire process. Thank you Dr. Dilika Peris, and Dr. Indika Perera for agreeing to serve on my evaluation panel and Dr. Srikantha Herath for agreeing to serve as my external supervisor.

I would like to acknowledge and thank Department of Computer Science and Engineering, University of Moratuwa, for allowing me to conduct my research and providing any assistance requested. Special thanks goes to both academic and non-academic staff of the department for their continued support. I also gratitude to the University of Moratuwa for the financial support as the research was supported in part by the Senate Research Grant of the University of Moratuwa under award number SRC/LT/2017/01.

Finally, I would like to thank the teachers, evaluating panel and colleagues that assisted me with this project. Their excitement and willingness to provide feedback made the completion of this research an enjoyable experience.

%%%%%%%%%%%%%%%%%%%%%%%%%%%%%%%%%%%%%%%%%%%%%%%%%%%%%%%%%%%%%%%%%%%%%%%%%%%%%%%%
\tableofcontents

\addcontentsline{toc}{chapter}{List of Figures}
\listoffigures

\addcontentsline{toc}{chapter}{List of Tables} 
\listoftables

\newglossarystyle{custom-glossary-style}{%
  \setglossarystyle{long}%
  \renewenvironment{theglossary}%
    {\begin{longtable}[l]{@{}p{\dimexpr 4cm-\tabcolsep}p{0.8\hsize}}}% <-- change the value here
    {\end{longtable}}%
}

\addcontentsline{toc}{chapter}{List of Abbreviations} 
\printglossary[style=custom-glossary-style, type=\acronymtype, title=List of Abbreviations, toctitle=List of Abbreviations, nonumberlist]

%%%%%%%%%%%%%%%%%%%%%%%%%%%%%%%%%%%%%%%%%%%%%%%%%%%%%%%%%%%%%%%%%%%%%%%%%%%%%%%%
\chapter{Introduction}
\label{ch:intro}
\pagenumbering{arabic}

\import{introduction/}{sec_1-motivation.tex}

\import{introduction/}{sec_2-problem_and_objectives.tex}

\import{introduction/}{sec_3-outline.tex}

%%%%%%%%%%%%%%%%%%%%%%%%%%%%%%%%%%%%%%%%%%%%%%%%%%%%%%%%%%%%%%%%%%%%%%%%%%%%%%%%
\chapter{Literature Review}
\label{ch:literature}
There are many of Weather forecasting, assimilation and dissemination systems developed by some of developed countries around the world as a result of trying to reduce the damage causing by natural disasters such as flood, storms, hurricanes, and even for drought. %Above kind of 
Such natural disasters heavily affected on the economy and the life conditions of the country, researchers have developed systems and many of these systems are using as proprietary systems. Other countries also using those systems and adopt with own version of system which effective for the specific weather condition in a particular country. 
%Though out 
This chapter presents a Literature review on existing systems and their architecture approach and design with available resources. \db{In Section 2.x we present ...} Most influenced systems for \acrshort{wdias} is \acrshort{fews} and \acrfull{lead} system. \acrshort{fews}, \db{an open data handling platform distributed as closed-source software, is presented in Section 2.x}. \db{In Section 2.x ... } \acrshort{lead} is following the Service Oriented Architecture for system architecture. Under each system we explore both system architecture, scalability and flexibility of the system.
\dbc{Complete above while referring to all major sections with chapter.}

\import{lit/}{sec_1-fews.tex}

\import{lit/}{sec_2-lead.tex}

\import{lit/}{sec_3-other-systems.tex}

%%%%%%%%%%%%%%%%%%%%%%%%%%%%%%%%%%%%%%%%%%%%%%%%%%%%%%%%%%%%%%%%%%%%%%%%%%%%%%%%
\chapter{Research Methodology}
\label{ch:method}

\import{method/}{sec_0_introduction.tex}
\label{se:method_intro}

\import{method/}{sec_1-soa.tex}
\label{se:soa}

\import{method/}{sec_2-actor_model.tex}
\label{se:actor_model}

\import{method/}{sec_3-microservice.tex}
\label{se:microservice}

\import{method/}{sec_4-db_struct.tex}
\label{se:db_struct}

\import{method/}{sec_5-data_preprocess.tex}
\label{se:data_preprocess}

\import{method/}{sec_6-query.tex}
\label{se:query}

%%%%%%%%%%%%%%%%%%%%%%%%%%%%%%%%%%%%%%%%%%%%%%%%%%%%%%%%%%%%%%%%%%%%%%%%%%%%%%%%
\chapter{Results}
\label{ch:results}

\import{results/}{sec_0-introduction.tex}
\label{se:results_intro}

\import{results/}{sec_1-test_plan.tex}
\label{se:test_plan}

\import{results/}{sec_2-workload.tex}
\label{se:workload}

\import{results/}{sec_3-observations.tex}
\label{se:observations}

\import{results/}{sec_4-discussion.tex}
\label{se:discussion}

%%%%%%%%%%%%%%%%%%%%%%%%%%%%%%%%%%%%%%%%%%%%%%%%%%%%%%%%%%%%%%%%%%%%%%%%%%%%%%%%
\chapter{\db{Summary}}
\label{ch:conc}

\import{summary/}{sec_1-conclusions.tex}

\import{summary/}{sec_2-research_limitations.tex}

%%%%%%%%%%%%%%%%%%%%%%%%%%%%%%%%%%%%%%%%%%%%%%%%%%%%%%%%%%%%%%%%%%%%%%%%%%%%%%%%
\appendix
\chapter{First Appendix}
Type out your first appendix here.
\section{Basics}

\chapter{First Appendix}
Type out your second appendix here.

\graphicspath{ {./images/} }
\newacronym{wdias}{WDIAS}{Weather Data Integration and Assimilation System}

\newacronym{fews}{Delft-FEWS} {Delft-FEWS, Deltares}
\newacronym{lead}{LEAD}{Linked Environments for Atmospheric Discovery}
\newacronym{dias}{DIAS}{Data Integration and Assimilation System}
\newacronym{madis}{MADIS}{Meteorological Assimilation Data Ingest System}

\newacronym{nwm}{NWMs}{Numerical Weather Models}
\newacronym{NetCDF}{NetCDF}{Network Common Data Form}
\newacronym{soa}{SOA}{Service Oriented Architecture}
\newacronym{wrf}{WRf}{Weather Research and Forecast}
\newacronym{esb}{ESB}{Enterprise Service Bus}
\newacronym{microservice}{Microservice}{Microservice Architectire}

\newacronym{curw}{CUrW SL}{Urban Center for Water, Sri Lanka}
% \bibliographystyle{plain}
% \addbibresource{mendeley.bib}
\printbibliography[title={References}]
\end{document}