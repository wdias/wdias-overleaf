\documentclass[a4paper,oneside,12pt]{report}
\usepackage{graphicx,mathptmx}
\usepackage{geometry}
 \geometry{
 a4paper,
 %total={160mm,227mm},
 left=40mm,
 top=25mm,
 bottom=40mm, 
 right=25mm,
}
\renewcommand{\baselinestretch}{1.5} 

\usepackage{setspace}
\usepackage{tabu}


\begin{document}
\begin{titlepage}
    \begin{center}
        \Large{
        \textbf{WEATHER DATA INTEGRATION AND ASSIMILATION SYSTEM}}\\
        \vspace{144pt}
  \large      
        %\vspace{1cm}
      % by
       
        
        %\vspace{1.0cm}
        
        Gihan Chanuka Karunarathne\\
        \vspace{24pt}
       % \normalsize
      178004U\\
         \vspace{72pt}
        %\normalsize
        Degree of Master of Science\\
       
        %\centering
         %\includegraphics[width=0.4\textwidth]{0.Title_Page/uom.png}\\
            %\vspace{0.8cm}
         %\normalsize
        
        %\vfill
       \vspace{72pt}
        \large
        Department of Computer Science and Engineering\\
        \vspace{24pt}
        University of Moratuwa\\
        Sri Lanka\\
        \vspace{32pt}
        April 2019
        
    \end{center}
\end{titlepage}

\begin{titlepage}
    \begin{center}
       % \vspace*{1cm}
        \Large{
        \textbf{WEATHER DATA INTEGRATION AND ASSIMILATION SYSTEM}}\\
        \vspace{144pt}
  \large      

        Herath Mudiyanselage Gihan Chanuka Karunarathne\\
        \vspace{24pt}
       % \normalsize
        178008U\\
         \vspace{72pt}
        \normalsize
        Thesis submitted in partial fulfillment of the requirements for the degree Master of Science in Computer Science and Engineering\\
     
        %\vfill
       \vspace{72pt}
        \large
        Department of Computer Science and Engineering\\
        \vspace{24pt}
        University of Moratuwa\\
        Sri Lanka\\
        \vspace{32pt}
        April 2019
        
    \end{center}
\end{titlepage}

\pagenumbering{roman}

\newgeometry{left=4cm,top=0.5cm,right=2.5cm}
\chapter*{Declaration}
\addcontentsline{toc}{chapter}{Declaration}

I declare that this is my own work and this thesis does not
incorporate without acknowledgement any material previously submitted for a
Degree or Diploma in any other University or institute of higher learning and to
the best of my knowledge and belief it does not contain any material previously
published or written by another person except where the acknowledgement is
made in the text.

Also, I hereby grant to University of Moratuwa the non-exclusive right to
reproduce and distribute my thesis/dissertation, in whole or in part in print,
electronic or other medium. I retain the right to use this content in whole or part
in future works (such as articles or books).

\vspace{0.5in}
\noindent
\begin{tabu} to 1.0\textwidth { X[l] X[l] }
    Signature: & Date:
\end{tabu}



\vspace{0.5in}
\noindent
The above candidate has carried out research for the Masters thesis under our supervision.


\vspace{0.5in}
\noindent
\begin{tabu} to 1.0\textwidth { X[l] X[l] }
    Name of the supervisor: & Dr. HMN Dilum Bandara\\ [1.5ex]
    Signature of the supervisor: & Date:
\end{tabu}

\restoregeometry
\normalsize


\newgeometry{left=2.5cm,top=0.2cm,right=2cm}
\addcontentsline{toc}{chapter}{Abstract} 
{\setstretch{1.0} 
\chapter*{Abstract}
%\chapter*{\hspace{1cm} Abstract}

To enhance the accuracy of weather forecast, it is necessary to provide reliable and detailed weather data as inputs to Numerical Weather Models (NWM). These NWMs utilize weather data collected via diverse sources such as automated weather stations, radars, air balloons, and satellite images. Prior to feeding such diverse data (collected from different sources that belongs to different stakeholders) into respective NWMs, it is necessary to integrate data into a common format. Moreover, the data integration system’s response time need to be relatively low to accommodate critical situations like hurricanes, storms, and floods which require rapid and frequent execution of NWMs.

Providing public to access weather data is also useful to enable many third-party applications and research. For example, logistic companies can use those data with their own model to plan and schedule their deliveries. Agricultural insurance companies can warn the farmers in advance, as well as calculate premiums based on anticipated weather patterns.

While Data Integration and Analysis System (DIAS) and Meteorological Assimilation Data Ingest System (MADIS) are some of well-known Weather Data Integration and Assimilation (WDIA) systems, they are proprietary. Further Delft-FEWS is free to use, it is not open source. Hence, users of WDIA are forced to pay heavy licenses and are unable to extend the solutions to cater to their country-specific requirements. Therefore, the objective of this research is to develop a WDIA system for Sri Lanka, as well as make it open source, so that others could user and contribute to the solution.

\vspace{4mm}

\textbf{Keywords:} WDIAS, Weather, Timeseries, Distributed, Scalable

}
\restoregeometry
\normalsize

\addcontentsline{toc}{chapter}{Dedication} 
\chapter*{Dedication}
I dedicate my thesis work to my family, teachers and specialy friends at CUrW. A special feeling of gratitude to my loving parents, Nandawathi Dissanayake and H.M.K. Karunarathne whose words of encouragement and push for tenacity ring in my ears. My wife Jayani Kumarasinghe have never left my side and are very special.

I also dedicate this thesis to my many friends at CUrW SL who have supported and be with me during this period of time. Also, special thanks to Dr. Dilum Bandara and Dr. Srikantha Herath for giving me this wonderful oppotunity to expore this new domain.

\addcontentsline{toc}{chapter}{Acknowledgements} 
\chapter*{Acknowledgements}
I wish to thank my evaluation panel members who were more than generous with their expertise and precious time. A special thanks to Dr. Dilum Bandara, my research supervisor for his countless hours of reflecting, reading, encouraging, and most of all patience throughout the entire process. Thank you Dr. Dilika Peris, and Dr. Indika Perera for agreeing to serve on my evaluation panel and Dr. Srikantha Herath for agreeing to serve as my external supervisor.

I would like to acknowledge and thank Department of Computer Science and Engineering, University of Moratuwa, for allowing me to conduct my research and providing any assistance requested. Special thanks goes to both academic and non-academic staff of the department for their continued support. I also gratitude to the University of Moratuwa for the financial support as the research was supported in part by the Senate Research Grant of the University of Moratuwa under award number SRC/LT/2017/01.

Finally, I would like to thank the teachers, evaluators and colleagues that assisted me with this project. Their excitement and willingness to provide feedback made the completion of this research an enjoyable experience.

\tableofcontents

\addcontentsline{toc}{chapter}{List of Figures} 
\listoffigures

\addcontentsline{toc}{chapter}{List of Tables} 
\listoftables

\addcontentsline{toc}{chapter}{List of Abbreviations} 
\chapter*{List of Abbreviations}

\pagenumbering{arabic}

\chapter{Introduction}
\label{ch:intro}

\section{Motivation}

\section{Problem}

\section{Objectives}

\section{Outline}

Introduce your thesis here.
\begin{figure}
\begin{center}
  \includegraphics[width=1in]{orchids.jpg}\\
  \caption{A picture of an orchid.}\label{fi:orchid}
\end{center}
\end{figure}
\cref{fi:orchid} shows an orchid.


\chapter{Literature Review}
\label{ch:literature}

Reviews literature and place your work within the existing body of literature here. In Chapter \ref{ch:intro} you introduced your thesis.

We can refer to books, journal articles, papers in conference proceeding etc. like this: Leslie Lamport \cite{Akka.ioWhenCluster} invented \LaTeX, based on \TeX.

\section{Delft-FEWS}

\section{LEAD}

\section{DIAS}

\section{MADIS}

\section{Summary}

\chapter{Method}
\label{ch:method}

Describe your method here so that the reader will be able to implement your system, if needed. 
\section{Introduction}
This is a section.
\label{se:method introduction}
Here is an equation:
\begin{equation}\label{eq:staright line}
    y = mx + c,
\end{equation}
where $m$ and $c$ are constants. Sometimes we use the form $ax + by + c = 0$, where $a$, $b$ and $c$ are constants. Equation \ref{eq:staright line} represents a straight line.

\subsection{Basics}
This is a subsection.

\section{Service Oriented Architecture}

\section{Microservice Architecture}

\section{Hierarchical Database Structure}

\section{Weather Data Prepossessing}

\subsection{Interpolation}

\subsection{Transformation}

\subsection{Validation}

\section{Query Timeseries}

\chapter{Results}
\label{ch:results}

Present your results here. 

Table \ref{ta:results} show some results.
\begin{table}
  \centering
  \begin{tabular}{|l|l|}
    \hline
    Day & Rainfall \\
    \hline
    Monday & 10\\
    Tuesday & 12\\
    Wednesday & 0\\
    \hline
  \end{tabular}
  \caption{Daily rainfall in millimeters.}\label{ta:results}
\end{table}

\section{Test Plan}

\section{Workload Creation}

\section{Observations}

\chapter{Conclusions}
\label{ch:summary}

Discuss your results and system and conclude.

\section{Summary}

\section{Future Work}

\appendix
\chapter{First Appendix}
Type out your first appendix here.
\section{Basics}

\chapter{First Appendix}
Type out your second appendix here.

% \bibliographystyle{plain}
% \bibliography{references}
% \end{document}

\bibliographystyle{plain}
\bibliography{mendeley}
\end{document}